\documentclass[DIV=9, pagesize=auto]{scrartcl}

\usepackage{fixltx2e}
\usepackage{lmodern}
\usepackage[T1]{fontenc}
\usepackage{microtype}
\usepackage{hyperref}

\newcommand*{\twoepsilon}{2$_{\textstyle\varepsilon}$}

\addtokomafont{title}{\rmfamily}

\title{The \textsf{bbm} package}
\author{Gilles F. Robert \\ \href{mailto:gfrobert@umpa.ens-lyon.fr}{gfrobert@umpa.ens-lyon.fr}}
\date{1999/03/15}


\begin{document}

\maketitle

\noindent
A mathematician often needs special symbols to designate sets such as the
integers, the reals etc..

For the time being, the only fonts providing these special symbols have been:
%
\begin{itemize}
\item the AMS fonts \textsf{msbm} (older version \textsf{msym}), which are intended for use with
  Times and aren't truly what a mathematician expects (the stress is on both
  stems instead of being on only the left one).
\item Alan Jeffrey's \textsf{bbold} fonts, which are sans-serif ones, with (almost) no
  possibility of design variations for, say, a bold variant.
\end{itemize}
%
That was all, and in particular nothing for use with Computer Modern.

I was already working at that time on the project of changing the codes of
\texttt{romanu.mf}, \texttt{romanl.mf} and other files to add a `Blackboard' variant to
Computer Modern.

\begin{center}
  YOU HAVE IT ALL THERE !!!  
\end{center}
%
The whole package contains five `main' files:
%
\begin{itemize}
\item \texttt{blbbase.mf} (the `base') contains the supplementary definitions needed
  for double stems etc..
\item \texttt{blbord.mf} (the `driver') is fairly simple and only makes the necessary
  calls.
\item \texttt{blbordu.mf} (based on \texttt{romanu.mf}): programs for uppercase letters.
\item \texttt{blbordl.mf} (based on \texttt{romanl.mf}): programs for lowercase letters.
\item \texttt{blbordsp.mf} (based on \texttt{romand.mf} and \texttt{punct.mf}): programs for the digits~1
and~2 and for parentheses and brackets.
\end{itemize}
%
There is also a whole lot of parameter files that are (almost) the same as
those provided by DEK.

For most of them, the only differences are
%
\begin{itemize}
\item change `\verb+cmbase+' for `\verb+blbbase+' on line~2
\item change `\verb+generate roman+' for `\verb+generate blbord+' on last line
\item add two extra parameters (\verb+interspace#+ and \verb+cap_interspace#+).
\end{itemize}

A \LaTeX(\twoepsilon) package written by Torsten Hilbrich for using these fonts is also 
available nearby, exactly in \href{http://mirror.ctan.org/macros/latex/contrib/supported/bbm/}{ctan/macros/latex/contrib/supported/bbm/}

I sincerely hope you'll enjoy using these fonts; if you get into problems
while using them, I should be reachable at
%
\begin{quote}
  Gilles F. ROBERT\\
  Unit\'e de Math\'ematiques Pures et Appliqu\'ees\\
  \'Ecole Normale Sup\'erieure\\
  46, All\'ee d'Italie\\
  69364 Lyon Cedex (FRANCE)\\
  \medskip
  e-mail~: \href{mailto:gfrobert@umpa.ens-lyon.fr}{gfrobert@umpa.ens-lyon.fr}
\end{quote}

\end{document}
