\documentclass[pagesize=auto, parskip=half, fontsize=12pt, DIV=13]{scrartcl}

\usepackage{fixltx2e}
\usepackage{etex}
\usepackage{lmodern}
\usepackage[T1]{fontenc}
\usepackage{textcomp}
\usepackage[svgnames]{xcolor}
\usepackage{hologo}
\usepackage{listings}
\usepackage{microtype}
\usepackage[breaklinks=true]{hyperref}

\newcommand*{\mail}[1]{\href{mailto:#1}{\texttt{#1}}}
\newcommand*{\bmail}[1]{\href{mailto:#1}{\texttt{<#1>}}}
\newcommand*{\pkg}[1]{\textsf{#1}}
\newcommand*{\cs}[1]{\texttt{\textbackslash#1}}
\makeatletter
\newcommand*{\cmd}[1]{\cs{\expandafter\@gobble\string#1}}
\makeatother
\newcommand*{\opt}[1]{\texttt{#1}}

\renewcommand*{\thesection}{\Alph{section}}

\addtokomafont{title}{\rmfamily}

\lstset{%
  language=[LaTeX]TeX,%
  columns=flexible,%
  upquote=true,%
  numbers=left,%
  basicstyle=\ttfamily,%
  keywordstyle=\color{Navy},%
  commentstyle=\color{DimGray},%
  stringstyle=\color{SeaGreen},%
  numberstyle=\scriptsize\color{SlateGray}%
}

\title{The \pkg{pax} package\thanks{This manual corresponds to \pkg{pax.sty}~v0.1h, dated~2007/07/19}}
\author{Heiko Oberdiek\thanks{oberdiek@uni-freiburg.de}}
\date{2007/07/19}


\begin{document}

\maketitle

\tableofcontents

\section{Project \pkg{pax}}

If PDF files are included using \hologo{pdfTeX}\@, PDF annotations
are stripped. This project tries a solution without
altering \hologo{pdfTeX}\@. A Java program (\texttt{pax.jar}) parses the PDF file that
will later be included. It writes the data of the annotations
in a file that can be read by \TeX\@. Then a \LaTeX\ package (\pkg{pax.sty})
extends the graphics package. If a PDF file is included, the package
looks for the file with the annotation data, reads them and puts
the annotations in the right place.

Project status: experimental


\section{Copyright, License}

Copyright~\textcopyright\ 2006--2008 Heiko Oberdiek.

The project consists of two parts:
%
\begin{itemize}
\item
  The Java program \texttt{pax.jar} with sources. \\
  The Perl wrapper \texttt{pdfannotextractor.pl}. \\
  License is GPL, see \texttt{license/gpl.txt}.
\item
  The \LaTeX\ package \pkg{pax.sty}.
  License is LPPL~1.3.
\end{itemize}
%
The file \texttt{README} belongs to both parts.

The Java program uses the PDF library PDFBox,
see \url{http://www.pdfbox.org/}.


\section{Files}

The project is available in CTAN (\href{ftp://ftp.ctan.org/}{\texttt{ftp.ctan.org}}): \\
\href{http://ctan.org/macros/latex/contrib/pax/}{\texttt{CTAN:macros/latex/contrib/pax/}}

There are the following files:
%
\begin{itemize}
\item
  \texttt{README} \\
  \null\quad This file.
\item
  \texttt{pax-tds.zip} \\
  \null\quad The whole project, ready to unpack in a TDS (\texttt{texmf}) tree.
\end{itemize}

The files inside pax-tds.zip:
%
\begin{itemize}
\item
  \texttt{doc/latex/pax/README} \\
  \null\quad This file.
\item
  \texttt{scripts/pax/pax.jar} \\
  \null\quad The Java program.
\item
  \texttt{scripts/pax/pdfannotextractor.pl} \\
  \null\quad Perl wrapper for calling the Java program in \texttt{pax.jar}.
\item
  \texttt{source/latex/pax/license/LaTeX/lppl.txt} \\
  \null\quad License (LPPL) for the \LaTeX\ package.
\item
  \texttt{source/latex/pax/license/PDFAnnotExtractor/gpl.txt} \\
  \null\quad License (GPL) for the Java program.
\item
  \begingroup
  \ttfamily
  source/latex/pax/src/Constants.java \\
  source/latex/pax/src/Entry.java \\
  source/latex/pax/src/EntryWriteException.java \\
  source/latex/pax/src/PDFAnnotExtractor.java \\
  source/latex/pax/src/StringVisitor.java
  \endgroup \\
  \null\quad Java source files.
\item
  \texttt{source/latex/pax/src/MANIFEST.MF} \\
  \null\quad Manifest file for the JAR file.
\item
  \texttt{source/latex/pax/src/build.xml} \\
  \null\quad Ant build file for generating the JAR file.
\item
  \texttt{tex/latex/pax/pax.sty} \\
  \null\quad The \LaTeX\ package.
\end{itemize}


\section{Author}

Heiko Oberdiek \bmail{oberdiek@uni-freiburg.de}


\section{Requirements}

\begin{itemize}
\item Java 1.4+.
\item PDFBox 0.7.2+.

\item The Perl wrapper (optional) needs:
  \begin{itemize}
  \item Perl
  \item \texttt{kpsewhich}
  \item \texttt{java} in \texttt{PATH}
  \end{itemize}

\item The option \opt{-{}-install} also requires:
  \begin{itemize}
  \item \texttt{wget} or \texttt{curl}
  \item \texttt{unzip}
  \item \texttt{texhash} or \texttt{mktexlsr}
  \end{itemize}
\end{itemize}


\section{Download}

\renewcommand*{\labelenumi}{\arabic{enumi}.}
\renewcommand*{\labelenumii}{\alph{enumii})}
\renewcommand*{\theenumi}{\labelenumi}
\renewcommand*{\theenumii}{\labelenumii}

\begin{enumerate}
\item \label{item:1}
  Download \texttt{pax-tds.zip} from CTAN: \\
  \url{ftp://ftp.ctan.org/tex-archive/macros/latex/contrib/pax/pax-tds.zip}

\item \label{item:2}
  Download \texttt{PDFBox-0.7.3.jar} from \url{http://www.pdfbox.org/} (22~MB) \\
  \url{http://prdownloads.sourceforge.net/pdfbox/PDFBox-0.7.3.zip?download} \\
  Or install PDFBox as RPM package or whatever.
\end{enumerate}


\section{Installation}

Examples are given for linux.

\begin{enumerate}
\setcounter{enumi}{2}
\item \label{item:3}
  Install PDFBox (from its home page, RPM package, \dots). \\
  The wrapper \texttt{pdfannotextractor.pl} looks for the JAR file in directories: \\
  \begingroup
  \ttfamily
  \null\quad /usr/share/java/ \\
  \null\quad /usr/local/share/java/ \\
  \null\quad TDS:scripts/pax/lib/ (TDS:scripts//) \\
  \endgroup
  assuming one of the following names: \\
  \begingroup
  \ttfamily
  \null\quad pdfbox.jar \\
  \null\quad PDFBox.jar \\
  \null\quad pdfbox-0.7.3.jar \\
  \null\quad PDFBox-0.7.3.jar \\
  \null\quad pdfbox-0.7.2.jar \\
  \null\quad PDFBox-0.7.2.jar
  \endgroup
\end{enumerate}

\minisec{Alternative for \ref{item:2} and \ref{item:3}}

\begin{itemize}
\item Continue with \ref{item:4} and \ref{item:5a}.
\item Call `\verb+pdfannotextractor --install+' (or with option \opt{-{}-debug}). \\
  It downloads PDFBox from its homepage and installs it
  in \texttt{TEXMFVAR(VARTEXMF)} below \texttt{TDS:scripts/pax/}
\end{itemize}

\begin{enumerate}
\setcounter{enumi}{3}
\item \label{item:4}
  Unzip \texttt{pax-tds.zip} inside the TDS tree, where you want to put this
   project, e.\,g.:
\begin{verbatim}
  unzip pax-tds -d /usr/local/share/texmf
\end{verbatim}
   Don't forget to update the database (\texttt{texhash}, \texttt{mktexlsr}, \dots).

 \item \label{item:5}
   \begin{enumerate}
   \item \label{item:5a}
     Install the wrapper Perl script \texttt{pdfannotextractor.pl}
     as `\texttt{pdfannotextractor'} somewhere in your \texttt{PATH} (\texttt{/usr/local/bin},
     \texttt{/usr/bin}, \dots).
   \item \label{item:5b}
     Or write a wrapper script or whatever to ease the call of the
     Java program, e.\,g.:
\begin{verbatim}
  #!/bin/sh
  java -cp pax.jar:pdfbox.jar pax.PDFAnnotExtractor "$@"
\end{verbatim}
   \end{enumerate}
\end{enumerate}


\subsection{Version of PDFBox}

Developing the \pkg{pax} project I have used and tested version~0.7.2,
also the current version~0.7.3 seems to be fine. Older versions
are not supported at all, they may work or do not.


\section{Use}

As example I am using \texttt{usrguide.pdf} (with links from project \texttt{latex-tds})
that will be included in \texttt{test.tex}:
%
\begin{lstlisting}
%%% cut %%% test.tex %%% cut %%%
\documentclass[a4paper,12pt,landscape]{article}
\usepackage[
  hmargin=1in,
  vmargin=1in,
]{geometry}

\usepackage{pdfpages}
% pdfpages loads graphicx

\usepackage{pax}

\iffalse
  \usepackage{hyperref}
  \hypersetup{
    filebordercolor={1 1 0},
  }
\fi

\begin{document}
  \includepdf[pages=-,nup=2x1]{usrguide}
\end{document}
%%% cut %%% test.tex %%% cut %%%
\end{lstlisting}

First run the Java program on \texttt{usrguide.pdf}:
%
\begin{verbatim}
  $ java -jar pax.jar usrguide.pdf
\end{verbatim}
%
It generates \texttt{usrguide.pax}.
Next run \texttt{pdflatex} on \texttt{test.tex} twice at least:
%
\begin{verbatim}
  $ pdflatex test
  $ pdflatex test
\end{verbatim}
%
Then the links should work.

Additionally package \pkg{hyperref} can be used (replace \verb+\iffase+ by \verb+\iftrue+).
Then the links can be configured using \pkg{hyperref} settings.
(But keep in mind, the colored links by option \opt{colorlinks} can't
be changed in included documents.


\section{History}

2006/08/24 v0.1a
%
\begin{itemize}
\item First release.
\end{itemize}

2006/09/04 v0.1b
%
\begin{itemize}
\item Bug fix in \verb+PDFAnnotExtractor.add_dest()+/named destinations.
\end{itemize}

2007/06/29 v0.1c
%
\begin{itemize}
\item \cmd{\l@addto@macro} renamed to \cmd{\PAX@l@addto@macro} to avoid
  collision with classes of \KOMAScript.
\end{itemize}

2007/06/30 v0.1d
%
\begin{itemize}
\item Use of package `\pkg{etexmcds}'.
\end{itemize}

2007/07/12 v0.1e
%
\begin{itemize}
\item The Java program can be called with several files.
\end{itemize}

2007/07/16 v0.1f
%
\begin{itemize}
\item PDF files are explicitely closed to avoid warnings.
\end{itemize}

2007/07/18 v0.1g
%
\begin{itemize}
\item Compiled for Java~1.4.
\end{itemize}

2007/07/19 v0.1h
%
\begin{itemize}
\item JAR file without TDS tree.
\end{itemize}

2008/10/01 v0.1i
%
\begin{itemize}
\item Perl wrapper `\texttt{pdfannotextractor.pl}' added.
\item Section `Installation' from \texttt{README} rewritten.
\item Class-Path removed from \texttt{MANIFEST.MF}.
\end{itemize}

\end{document}
