\documentclass[pagesize=auto, fontsize=12pt]{scrartcl}

\usepackage{fixltx2e}
\usepackage{etex}
\usepackage{lmodern}
\usepackage[T1]{fontenc}
\usepackage{textcomp}
\usepackage{hologo}
\usepackage{microtype}
\usepackage{hyperref}

\newcommand*{\mail}[1]{\href{mailto:#1}{\texttt{#1}}}
\newcommand*{\pkg}[1]{\textsf{#1}}
\newcommand*{\cs}[1]{\texttt{\textbackslash#1}}
\makeatletter
\newcommand*{\cmd}[1]{\cs{\expandafter\@gobble\string#1}}
\makeatother
\newcommand*{\opt}[1]{\texttt{#1}}
\newcommand*{\meta}[1]{\textlangle\textsl{#1}\textrangle}
\newcommand*{\marg}[1]{\texttt{\{}\meta{#1}\texttt{\}}}
\newcommand*{\outputonly}[1]{\cmd{\outputonly}\texttt{\{#1\}}}

\addtokomafont{title}{\rmfamily}

\title{The \pkg{selectp} package\thanks{This manual corresponds to \pkg{selectp}~v1.0, dated~Sept 25, 1992.}}
\subtitle{Select pages for output}
\author{%
  Donald Arseneau\thanks{\mail{asnd@triumf.ca}},%
  \and based on macros in TUGBoat, 8:2 (1987), p.~217%
  \and written by Don Knuth,%
  \and and with suggestions from Hossein Saiedian.%
}
\date{Sept 25, 1992}


\begin{document}

\maketitle


\section{Instructions}

This style file defines the command \cmd{\outputonly} which selects specific
pages for output, much as \cmd{\includeonly} selects certain files for input.
To use \pkg{selectp.sty} in \LaTeX, specify \opt{selectp} as one of the document
style options and give the command \cmd{\outputonly}\marg{list of page numbers}
before \verb|\begin{document}|.  Only pages given in the list will be output
to the DVI file.  For example
%
\begin{verbatim}
    \documentstyle[12pt,selectp,subeqn]{article}
    \outputonly{1,3, 7-12 16 17}
\end{verbatim}
%
which will allow only pages 1, 3, 7, 8, 9, 10, 11, 12, 16, 17 to be output.
For \hologo{plainTeX}, use \verb|\documentclass[pagesize=auto, fontsize=12pt]{scrartcl}

\usepackage{fixltx2e}
\usepackage{etex}
\usepackage{lmodern}
\usepackage[T1]{fontenc}
\usepackage{textcomp}
\usepackage{hologo}
\usepackage{microtype}
\usepackage{hyperref}

\newcommand*{\mail}[1]{\href{mailto:#1}{\texttt{#1}}}
\newcommand*{\pkg}[1]{\textsf{#1}}
\newcommand*{\cs}[1]{\texttt{\textbackslash#1}}
\makeatletter
\newcommand*{\cmd}[1]{\cs{\expandafter\@gobble\string#1}}
\makeatother
\newcommand*{\opt}[1]{\texttt{#1}}
\newcommand*{\meta}[1]{\textlangle\textsl{#1}\textrangle}
\newcommand*{\marg}[1]{\texttt{\{}\meta{#1}\texttt{\}}}
\newcommand*{\outputonly}[1]{\cmd{\outputonly}\texttt{\{#1\}}}

\addtokomafont{title}{\rmfamily}

\title{The \pkg{selectp} package\thanks{This manual corresponds to \pkg{selectp}~v1.0, dated~Sept 25, 1992.}}
\subtitle{Select pages for output}
\author{%
  Donald Arseneau\thanks{\mail{asnd@triumf.ca}},%
  \and based on macros in TUGBoat, 8:2 (1987), p.~217%
  \and written by Don Knuth,%
  \and and with suggestions from Hossein Saiedian.%
}
\date{Sept 25, 1992}


\begin{document}

\maketitle


\section{Instructions}

This style file defines the command \cmd{\outputonly} which selects specific
pages for output, much as \cmd{\includeonly} selects certain files for input.
To use \pkg{selectp.sty} in \LaTeX, specify \opt{selectp} as one of the document
style options and give the command \cmd{\outputonly}\marg{list of page numbers}
before \verb|\begin{document}|.  Only pages given in the list will be output
to the DVI file.  For example
%
\begin{verbatim}
    \documentstyle[12pt,selectp,subeqn]{article}
    \outputonly{1,3, 7-12 16 17}
\end{verbatim}
%
which will allow only pages 1, 3, 7, 8, 9, 10, 11, 12, 16, 17 to be output.
For \hologo{plainTeX}, use \verb|\documentclass[pagesize=auto, fontsize=12pt]{scrartcl}

\usepackage{fixltx2e}
\usepackage{etex}
\usepackage{lmodern}
\usepackage[T1]{fontenc}
\usepackage{textcomp}
\usepackage{hologo}
\usepackage{microtype}
\usepackage{hyperref}

\newcommand*{\mail}[1]{\href{mailto:#1}{\texttt{#1}}}
\newcommand*{\pkg}[1]{\textsf{#1}}
\newcommand*{\cs}[1]{\texttt{\textbackslash#1}}
\makeatletter
\newcommand*{\cmd}[1]{\cs{\expandafter\@gobble\string#1}}
\makeatother
\newcommand*{\opt}[1]{\texttt{#1}}
\newcommand*{\meta}[1]{\textlangle\textsl{#1}\textrangle}
\newcommand*{\marg}[1]{\texttt{\{}\meta{#1}\texttt{\}}}
\newcommand*{\outputonly}[1]{\cmd{\outputonly}\texttt{\{#1\}}}

\addtokomafont{title}{\rmfamily}

\title{The \pkg{selectp} package\thanks{This manual corresponds to \pkg{selectp}~v1.0, dated~Sept 25, 1992.}}
\subtitle{Select pages for output}
\author{%
  Donald Arseneau\thanks{\mail{asnd@triumf.ca}},%
  \and based on macros in TUGBoat, 8:2 (1987), p.~217%
  \and written by Don Knuth,%
  \and and with suggestions from Hossein Saiedian.%
}
\date{Sept 25, 1992}


\begin{document}

\maketitle


\section{Instructions}

This style file defines the command \cmd{\outputonly} which selects specific
pages for output, much as \cmd{\includeonly} selects certain files for input.
To use \pkg{selectp.sty} in \LaTeX, specify \opt{selectp} as one of the document
style options and give the command \cmd{\outputonly}\marg{list of page numbers}
before \verb|\begin{document}|.  Only pages given in the list will be output
to the DVI file.  For example
%
\begin{verbatim}
    \documentstyle[12pt,selectp,subeqn]{article}
    \outputonly{1,3, 7-12 16 17}
\end{verbatim}
%
which will allow only pages 1, 3, 7, 8, 9, 10, 11, 12, 16, 17 to be output.
For \hologo{plainTeX}, use \verb|\documentclass[pagesize=auto, fontsize=12pt]{scrartcl}

\usepackage{fixltx2e}
\usepackage{etex}
\usepackage{lmodern}
\usepackage[T1]{fontenc}
\usepackage{textcomp}
\usepackage{hologo}
\usepackage{microtype}
\usepackage{hyperref}

\newcommand*{\mail}[1]{\href{mailto:#1}{\texttt{#1}}}
\newcommand*{\pkg}[1]{\textsf{#1}}
\newcommand*{\cs}[1]{\texttt{\textbackslash#1}}
\makeatletter
\newcommand*{\cmd}[1]{\cs{\expandafter\@gobble\string#1}}
\makeatother
\newcommand*{\opt}[1]{\texttt{#1}}
\newcommand*{\meta}[1]{\textlangle\textsl{#1}\textrangle}
\newcommand*{\marg}[1]{\texttt{\{}\meta{#1}\texttt{\}}}
\newcommand*{\outputonly}[1]{\cmd{\outputonly}\texttt{\{#1\}}}

\addtokomafont{title}{\rmfamily}

\title{The \pkg{selectp} package\thanks{This manual corresponds to \pkg{selectp}~v1.0, dated~Sept 25, 1992.}}
\subtitle{Select pages for output}
\author{%
  Donald Arseneau\thanks{\mail{asnd@triumf.ca}},%
  \and based on macros in TUGBoat, 8:2 (1987), p.~217%
  \and written by Don Knuth,%
  \and and with suggestions from Hossein Saiedian.%
}
\date{Sept 25, 1992}


\begin{document}

\maketitle


\section{Instructions}

This style file defines the command \cmd{\outputonly} which selects specific
pages for output, much as \cmd{\includeonly} selects certain files for input.
To use \pkg{selectp.sty} in \LaTeX, specify \opt{selectp} as one of the document
style options and give the command \cmd{\outputonly}\marg{list of page numbers}
before \verb|\begin{document}|.  Only pages given in the list will be output
to the DVI file.  For example
%
\begin{verbatim}
    \documentstyle[12pt,selectp,subeqn]{article}
    \outputonly{1,3, 7-12 16 17}
\end{verbatim}
%
which will allow only pages 1, 3, 7, 8, 9, 10, 11, 12, 16, 17 to be output.
For \hologo{plainTeX}, use \verb|\input{selectp.sty}|.

The number list should consist of numbers and number ranges (\verb|7-12|, e.\,g.)\ %
separated by commas or spaces. The order is not random but \emph{must be the
order that the pages are produced}; normally this means the numbers must
always increase.  If an invalid page number appears on the list, no pages
will be output until the list is re-synchronized or the document ends.

Any time the document's page number does not increment normally,
\pkg{selectp.sty} tries to re-synchronize the page number with the \cmd{\outputonly}
list of numbers.  This is necessary when \cmd{\includeonly} is used. 

Pages that are labeled with roman numerals or letters should still be
listed with a normal arabic number; page~xi should be referred to as~6.

In \LaTeX, unfortunately, pages~i and~1 are both number~1, so it is
tricky to skip over all the roman-numbered pages and then print
page~1.  If there are 3 roman-numbered pages,
%
\begingroup
\addtokomafont{labelinglabel}{\ttfamily}
\begin{labeling}{1-3,8-12}
\item[1-3]
  will print pages i, ii, iii; not pages 1, 2, 3
\item[1-3,1-5]
  will print pages i, ii, iii, 1, 2, 3, 4, 5
\item[1-3,8-12]
  will print pages i, ii, iii, 8, 9, 10, 11, 19
\item[1,1-5]
  will print i, 1, 2, 3, 4, 5
\item[1-6]
  prints pages i, ii, iii only, not pages 4, 5, 6  !!!
\item[0,2-6]
  prints pages 2, 3, 4, 5, 6, since there is no page~0 and the
  \verb|outputonly| page list is resynchronized when the actual page
  number goes back to~1)
\end{labeling}
\endgroup
%
The last example is illuminating.  \LaTeX\ looks for page~0, but can't find
it. It skips pages until the page number goes from~3~(iii) to~1, at which
point it resynchronizes by reading~\texttt{2} from the list, waiting for page~2
to be produced, and then writing page~2. Selectp then reads~\texttt{-6}, and
\emph{Continues} outputting pages until it has done page~6.  As long as the
numbers of the intervening pages are less than~6 (the end of the range),
those pages will be printed. 

If there are no roman numeral pages, the number list is simple, because
the numbers must increase monotonically.

\LaTeX\ will not write auxilliary files while selecting output pages, so
the cross references and citations must be correct on the run \emph{before} 
using \cmd{\outputonly}.  Unfortunately, this means you must produce the 
full-size DVI file at least once.

If multuple \cmd{\outputonly} commands are given, the lists are concatenated.

In \hologo{plainTeX}, specify roman numeral pages as negative numbers, but do not
use ranges until getting to the ordinary (arabic) page numbers (e.\,g.,\ %
\outputonly{-1,-2,-3,-4, 1-8}).  In \hologo{plainTeX} specifying page~1 will \emph{not}
print page~i. 

If a page number in the list is not found, generally no more pages will be
printed. For example, suppose a document has 50~pages, the list \verb*|3 5 910 13|
(which perhaps should read \verb*|9 10|) causes \LaTeX\ to process silently
through the whole document looking for page~910.  Only pages~3 and~5 will
be printed. 

\emph{Except} if the number terminating a \emph{range} is not found \verb*|{3 5-910 13}|,
\LaTeX\ continues outputting pages until the end.

\emph{Except} if the actual page number changes discontinuously, then the number
list \emph{may} be scanned for the next number greater than the new page number.
(Jumps are usually due to the numbering being changed from roman to arabic
with the page reset to~1, or because an included file was skipped due to
an \cmd{\includeonly} command.) The list is scanned if the page number jumps
backward, or if it jumps forward \emph{past} the current target page. (``Target
page'' = the end of the current range or the next number to be printed out,
as appropriate.) 

Numbers must be integers: \texttt{4.3} is illegal, even if some other style file
is generating page numbers in that format.

\pagebreak[1]

Of course non-numbers are illegal.  Some plausible mistakes:
%
\begin{quote}
  \verb|page|, \verb|xvii|, \verb|IV|, \verb|12--16|, \verb|49ff|, \verb|49-|, \verb|5_7|.  
\end{quote}
%
\verb|5_7| could be a typo for the range \verb|5-7|.
To specify page~49 and all following (\verb|49ff| or \verb|49-|), use a range 
ending with a non-existent page:  \verb|49-99999|.
The typo \verb|12--16| prints pages~12 and~16, not the range~12 through~16 as
intended.


\section{Examples of use}

Suppose a document has pages i, ii, iii, iv, v, vi, 1, 2, 3,\dots, 33, but pages
22\dots28 are absent due to \cmd{\includeonly}; here are some examples of valid
uses of \cmd{\outputonly}.
%
\begin{labeling}[\enspace\textendash]{\outputonly{0,4,23-27,31-99}}
\item[\outputonly{15-20}]
  prints only the 6 pages 15...20
\item[\outputonly{1-3}]
  prints pages i, ii, iii (in \LaTeX)\\
  \hologo{plainTeX} prints pages 1, 2, 3
\item[\outputonly{1,1,2}]
  prints pages i, 1, 2
\item[\outputonly{0,1,2}]
  prints pages 1, 2
\item[\outputonly{1,3-6}]
  prints pages i, iii, iv, v, vi
\item[\outputonly{4,3-6}]
  prints pages iv, 3, 4, 5, 6
\item[\outputonly{5,2}]
  prints pages v, 2
\item[\outputonly{5-2}]
  prints only page v    only!
\item[\outputonly{5-7}]
  prints pages v, vi     only!
\item[\outputonly{1-33}]
  prints pages i, ii, iii, iv, v, vi  only!
\item[\outputonly{29-99}]
  prints pages 29, 30, 31, 32, 33
\item[\outputonly{19-99}]
  prints pages 19, 20, 21, 29, 30, 31, 32, 33
\item[\outputonly{24-32}]
  prints pages 29, 30, 31, 32
\item[\outputonly{15-25,32}]
  prints pages 15, 16, 17, 18, 19, 20, 21, 32
\item[\outputonly{0,4,23-27,31-99}]
  prints pages 4, 31, 32, 33
\end{labeling}

Version 0.9 (test), Sept 25, 1992:
Send problem reports to \mail{asnd@triumf.ca}

\end{document}
|.

The number list should consist of numbers and number ranges (\verb|7-12|, e.\,g.)\ %
separated by commas or spaces. The order is not random but \emph{must be the
order that the pages are produced}; normally this means the numbers must
always increase.  If an invalid page number appears on the list, no pages
will be output until the list is re-synchronized or the document ends.

Any time the document's page number does not increment normally,
\pkg{selectp.sty} tries to re-synchronize the page number with the \cmd{\outputonly}
list of numbers.  This is necessary when \cmd{\includeonly} is used. 

Pages that are labeled with roman numerals or letters should still be
listed with a normal arabic number; page~xi should be referred to as~6.

In \LaTeX, unfortunately, pages~i and~1 are both number~1, so it is
tricky to skip over all the roman-numbered pages and then print
page~1.  If there are 3 roman-numbered pages,
%
\begingroup
\addtokomafont{labelinglabel}{\ttfamily}
\begin{labeling}{1-3,8-12}
\item[1-3]
  will print pages i, ii, iii; not pages 1, 2, 3
\item[1-3,1-5]
  will print pages i, ii, iii, 1, 2, 3, 4, 5
\item[1-3,8-12]
  will print pages i, ii, iii, 8, 9, 10, 11, 19
\item[1,1-5]
  will print i, 1, 2, 3, 4, 5
\item[1-6]
  prints pages i, ii, iii only, not pages 4, 5, 6  !!!
\item[0,2-6]
  prints pages 2, 3, 4, 5, 6, since there is no page~0 and the
  \verb|outputonly| page list is resynchronized when the actual page
  number goes back to~1)
\end{labeling}
\endgroup
%
The last example is illuminating.  \LaTeX\ looks for page~0, but can't find
it. It skips pages until the page number goes from~3~(iii) to~1, at which
point it resynchronizes by reading~\texttt{2} from the list, waiting for page~2
to be produced, and then writing page~2. Selectp then reads~\texttt{-6}, and
\emph{Continues} outputting pages until it has done page~6.  As long as the
numbers of the intervening pages are less than~6 (the end of the range),
those pages will be printed. 

If there are no roman numeral pages, the number list is simple, because
the numbers must increase monotonically.

\LaTeX\ will not write auxilliary files while selecting output pages, so
the cross references and citations must be correct on the run \emph{before} 
using \cmd{\outputonly}.  Unfortunately, this means you must produce the 
full-size DVI file at least once.

If multuple \cmd{\outputonly} commands are given, the lists are concatenated.

In \hologo{plainTeX}, specify roman numeral pages as negative numbers, but do not
use ranges until getting to the ordinary (arabic) page numbers (e.\,g.,\ %
\outputonly{-1,-2,-3,-4, 1-8}).  In \hologo{plainTeX} specifying page~1 will \emph{not}
print page~i. 

If a page number in the list is not found, generally no more pages will be
printed. For example, suppose a document has 50~pages, the list \verb*|3 5 910 13|
(which perhaps should read \verb*|9 10|) causes \LaTeX\ to process silently
through the whole document looking for page~910.  Only pages~3 and~5 will
be printed. 

\emph{Except} if the number terminating a \emph{range} is not found \verb*|{3 5-910 13}|,
\LaTeX\ continues outputting pages until the end.

\emph{Except} if the actual page number changes discontinuously, then the number
list \emph{may} be scanned for the next number greater than the new page number.
(Jumps are usually due to the numbering being changed from roman to arabic
with the page reset to~1, or because an included file was skipped due to
an \cmd{\includeonly} command.) The list is scanned if the page number jumps
backward, or if it jumps forward \emph{past} the current target page. (``Target
page'' = the end of the current range or the next number to be printed out,
as appropriate.) 

Numbers must be integers: \texttt{4.3} is illegal, even if some other style file
is generating page numbers in that format.

\pagebreak[1]

Of course non-numbers are illegal.  Some plausible mistakes:
%
\begin{quote}
  \verb|page|, \verb|xvii|, \verb|IV|, \verb|12--16|, \verb|49ff|, \verb|49-|, \verb|5_7|.  
\end{quote}
%
\verb|5_7| could be a typo for the range \verb|5-7|.
To specify page~49 and all following (\verb|49ff| or \verb|49-|), use a range 
ending with a non-existent page:  \verb|49-99999|.
The typo \verb|12--16| prints pages~12 and~16, not the range~12 through~16 as
intended.


\section{Examples of use}

Suppose a document has pages i, ii, iii, iv, v, vi, 1, 2, 3,\dots, 33, but pages
22\dots28 are absent due to \cmd{\includeonly}; here are some examples of valid
uses of \cmd{\outputonly}.
%
\begin{labeling}[\enspace\textendash]{\outputonly{0,4,23-27,31-99}}
\item[\outputonly{15-20}]
  prints only the 6 pages 15...20
\item[\outputonly{1-3}]
  prints pages i, ii, iii (in \LaTeX)\\
  \hologo{plainTeX} prints pages 1, 2, 3
\item[\outputonly{1,1,2}]
  prints pages i, 1, 2
\item[\outputonly{0,1,2}]
  prints pages 1, 2
\item[\outputonly{1,3-6}]
  prints pages i, iii, iv, v, vi
\item[\outputonly{4,3-6}]
  prints pages iv, 3, 4, 5, 6
\item[\outputonly{5,2}]
  prints pages v, 2
\item[\outputonly{5-2}]
  prints only page v    only!
\item[\outputonly{5-7}]
  prints pages v, vi     only!
\item[\outputonly{1-33}]
  prints pages i, ii, iii, iv, v, vi  only!
\item[\outputonly{29-99}]
  prints pages 29, 30, 31, 32, 33
\item[\outputonly{19-99}]
  prints pages 19, 20, 21, 29, 30, 31, 32, 33
\item[\outputonly{24-32}]
  prints pages 29, 30, 31, 32
\item[\outputonly{15-25,32}]
  prints pages 15, 16, 17, 18, 19, 20, 21, 32
\item[\outputonly{0,4,23-27,31-99}]
  prints pages 4, 31, 32, 33
\end{labeling}

Version 0.9 (test), Sept 25, 1992:
Send problem reports to \mail{asnd@triumf.ca}

\end{document}
|.

The number list should consist of numbers and number ranges (\verb|7-12|, e.\,g.)\ %
separated by commas or spaces. The order is not random but \emph{must be the
order that the pages are produced}; normally this means the numbers must
always increase.  If an invalid page number appears on the list, no pages
will be output until the list is re-synchronized or the document ends.

Any time the document's page number does not increment normally,
\pkg{selectp.sty} tries to re-synchronize the page number with the \cmd{\outputonly}
list of numbers.  This is necessary when \cmd{\includeonly} is used. 

Pages that are labeled with roman numerals or letters should still be
listed with a normal arabic number; page~xi should be referred to as~6.

In \LaTeX, unfortunately, pages~i and~1 are both number~1, so it is
tricky to skip over all the roman-numbered pages and then print
page~1.  If there are 3 roman-numbered pages,
%
\begingroup
\addtokomafont{labelinglabel}{\ttfamily}
\begin{labeling}{1-3,8-12}
\item[1-3]
  will print pages i, ii, iii; not pages 1, 2, 3
\item[1-3,1-5]
  will print pages i, ii, iii, 1, 2, 3, 4, 5
\item[1-3,8-12]
  will print pages i, ii, iii, 8, 9, 10, 11, 19
\item[1,1-5]
  will print i, 1, 2, 3, 4, 5
\item[1-6]
  prints pages i, ii, iii only, not pages 4, 5, 6  !!!
\item[0,2-6]
  prints pages 2, 3, 4, 5, 6, since there is no page~0 and the
  \verb|outputonly| page list is resynchronized when the actual page
  number goes back to~1)
\end{labeling}
\endgroup
%
The last example is illuminating.  \LaTeX\ looks for page~0, but can't find
it. It skips pages until the page number goes from~3~(iii) to~1, at which
point it resynchronizes by reading~\texttt{2} from the list, waiting for page~2
to be produced, and then writing page~2. Selectp then reads~\texttt{-6}, and
\emph{Continues} outputting pages until it has done page~6.  As long as the
numbers of the intervening pages are less than~6 (the end of the range),
those pages will be printed. 

If there are no roman numeral pages, the number list is simple, because
the numbers must increase monotonically.

\LaTeX\ will not write auxilliary files while selecting output pages, so
the cross references and citations must be correct on the run \emph{before} 
using \cmd{\outputonly}.  Unfortunately, this means you must produce the 
full-size DVI file at least once.

If multuple \cmd{\outputonly} commands are given, the lists are concatenated.

In \hologo{plainTeX}, specify roman numeral pages as negative numbers, but do not
use ranges until getting to the ordinary (arabic) page numbers (e.\,g.,\ %
\outputonly{-1,-2,-3,-4, 1-8}).  In \hologo{plainTeX} specifying page~1 will \emph{not}
print page~i. 

If a page number in the list is not found, generally no more pages will be
printed. For example, suppose a document has 50~pages, the list \verb*|3 5 910 13|
(which perhaps should read \verb*|9 10|) causes \LaTeX\ to process silently
through the whole document looking for page~910.  Only pages~3 and~5 will
be printed. 

\emph{Except} if the number terminating a \emph{range} is not found \verb*|{3 5-910 13}|,
\LaTeX\ continues outputting pages until the end.

\emph{Except} if the actual page number changes discontinuously, then the number
list \emph{may} be scanned for the next number greater than the new page number.
(Jumps are usually due to the numbering being changed from roman to arabic
with the page reset to~1, or because an included file was skipped due to
an \cmd{\includeonly} command.) The list is scanned if the page number jumps
backward, or if it jumps forward \emph{past} the current target page. (``Target
page'' = the end of the current range or the next number to be printed out,
as appropriate.) 

Numbers must be integers: \texttt{4.3} is illegal, even if some other style file
is generating page numbers in that format.

\pagebreak[1]

Of course non-numbers are illegal.  Some plausible mistakes:
%
\begin{quote}
  \verb|page|, \verb|xvii|, \verb|IV|, \verb|12--16|, \verb|49ff|, \verb|49-|, \verb|5_7|.  
\end{quote}
%
\verb|5_7| could be a typo for the range \verb|5-7|.
To specify page~49 and all following (\verb|49ff| or \verb|49-|), use a range 
ending with a non-existent page:  \verb|49-99999|.
The typo \verb|12--16| prints pages~12 and~16, not the range~12 through~16 as
intended.


\section{Examples of use}

Suppose a document has pages i, ii, iii, iv, v, vi, 1, 2, 3,\dots, 33, but pages
22\dots28 are absent due to \cmd{\includeonly}; here are some examples of valid
uses of \cmd{\outputonly}.
%
\begin{labeling}[\enspace\textendash]{\outputonly{0,4,23-27,31-99}}
\item[\outputonly{15-20}]
  prints only the 6 pages 15...20
\item[\outputonly{1-3}]
  prints pages i, ii, iii (in \LaTeX)\\
  \hologo{plainTeX} prints pages 1, 2, 3
\item[\outputonly{1,1,2}]
  prints pages i, 1, 2
\item[\outputonly{0,1,2}]
  prints pages 1, 2
\item[\outputonly{1,3-6}]
  prints pages i, iii, iv, v, vi
\item[\outputonly{4,3-6}]
  prints pages iv, 3, 4, 5, 6
\item[\outputonly{5,2}]
  prints pages v, 2
\item[\outputonly{5-2}]
  prints only page v    only!
\item[\outputonly{5-7}]
  prints pages v, vi     only!
\item[\outputonly{1-33}]
  prints pages i, ii, iii, iv, v, vi  only!
\item[\outputonly{29-99}]
  prints pages 29, 30, 31, 32, 33
\item[\outputonly{19-99}]
  prints pages 19, 20, 21, 29, 30, 31, 32, 33
\item[\outputonly{24-32}]
  prints pages 29, 30, 31, 32
\item[\outputonly{15-25,32}]
  prints pages 15, 16, 17, 18, 19, 20, 21, 32
\item[\outputonly{0,4,23-27,31-99}]
  prints pages 4, 31, 32, 33
\end{labeling}

Version 0.9 (test), Sept 25, 1992:
Send problem reports to \mail{asnd@triumf.ca}

\end{document}
|.

The number list should consist of numbers and number ranges (\verb|7-12|, e.\,g.)\ %
separated by commas or spaces. The order is not random but \emph{must be the
order that the pages are produced}; normally this means the numbers must
always increase.  If an invalid page number appears on the list, no pages
will be output until the list is re-synchronized or the document ends.

Any time the document's page number does not increment normally,
\pkg{selectp.sty} tries to re-synchronize the page number with the \cmd{\outputonly}
list of numbers.  This is necessary when \cmd{\includeonly} is used. 

Pages that are labeled with roman numerals or letters should still be
listed with a normal arabic number; page~xi should be referred to as~6.

In \LaTeX, unfortunately, pages~i and~1 are both number~1, so it is
tricky to skip over all the roman-numbered pages and then print
page~1.  If there are 3 roman-numbered pages,
%
\begingroup
\addtokomafont{labelinglabel}{\ttfamily}
\begin{labeling}{1-3,8-12}
\item[1-3]
  will print pages i, ii, iii; not pages 1, 2, 3
\item[1-3,1-5]
  will print pages i, ii, iii, 1, 2, 3, 4, 5
\item[1-3,8-12]
  will print pages i, ii, iii, 8, 9, 10, 11, 19
\item[1,1-5]
  will print i, 1, 2, 3, 4, 5
\item[1-6]
  prints pages i, ii, iii only, not pages 4, 5, 6  !!!
\item[0,2-6]
  prints pages 2, 3, 4, 5, 6, since there is no page~0 and the
  \verb|outputonly| page list is resynchronized when the actual page
  number goes back to~1)
\end{labeling}
\endgroup
%
The last example is illuminating.  \LaTeX\ looks for page~0, but can't find
it. It skips pages until the page number goes from~3~(iii) to~1, at which
point it resynchronizes by reading~\texttt{2} from the list, waiting for page~2
to be produced, and then writing page~2. Selectp then reads~\texttt{-6}, and
\emph{Continues} outputting pages until it has done page~6.  As long as the
numbers of the intervening pages are less than~6 (the end of the range),
those pages will be printed. 

If there are no roman numeral pages, the number list is simple, because
the numbers must increase monotonically.

\LaTeX\ will not write auxilliary files while selecting output pages, so
the cross references and citations must be correct on the run \emph{before} 
using \cmd{\outputonly}.  Unfortunately, this means you must produce the 
full-size DVI file at least once.

If multuple \cmd{\outputonly} commands are given, the lists are concatenated.

In \hologo{plainTeX}, specify roman numeral pages as negative numbers, but do not
use ranges until getting to the ordinary (arabic) page numbers (e.\,g.,\ %
\outputonly{-1,-2,-3,-4, 1-8}).  In \hologo{plainTeX} specifying page~1 will \emph{not}
print page~i. 

If a page number in the list is not found, generally no more pages will be
printed. For example, suppose a document has 50~pages, the list \verb*|3 5 910 13|
(which perhaps should read \verb*|9 10|) causes \LaTeX\ to process silently
through the whole document looking for page~910.  Only pages~3 and~5 will
be printed. 

\emph{Except} if the number terminating a \emph{range} is not found \verb*|{3 5-910 13}|,
\LaTeX\ continues outputting pages until the end.

\emph{Except} if the actual page number changes discontinuously, then the number
list \emph{may} be scanned for the next number greater than the new page number.
(Jumps are usually due to the numbering being changed from roman to arabic
with the page reset to~1, or because an included file was skipped due to
an \cmd{\includeonly} command.) The list is scanned if the page number jumps
backward, or if it jumps forward \emph{past} the current target page. (``Target
page'' = the end of the current range or the next number to be printed out,
as appropriate.) 

Numbers must be integers: \texttt{4.3} is illegal, even if some other style file
is generating page numbers in that format.

\pagebreak[1]

Of course non-numbers are illegal.  Some plausible mistakes:
%
\begin{quote}
  \verb|page|, \verb|xvii|, \verb|IV|, \verb|12--16|, \verb|49ff|, \verb|49-|, \verb|5_7|.  
\end{quote}
%
\verb|5_7| could be a typo for the range \verb|5-7|.
To specify page~49 and all following (\verb|49ff| or \verb|49-|), use a range 
ending with a non-existent page:  \verb|49-99999|.
The typo \verb|12--16| prints pages~12 and~16, not the range~12 through~16 as
intended.


\section{Examples of use}

Suppose a document has pages i, ii, iii, iv, v, vi, 1, 2, 3,\dots, 33, but pages
22\dots28 are absent due to \cmd{\includeonly}; here are some examples of valid
uses of \cmd{\outputonly}.
%
\begin{labeling}[\enspace\textendash]{\outputonly{0,4,23-27,31-99}}
\item[\outputonly{15-20}]
  prints only the 6 pages 15...20
\item[\outputonly{1-3}]
  prints pages i, ii, iii (in \LaTeX)\\
  \hologo{plainTeX} prints pages 1, 2, 3
\item[\outputonly{1,1,2}]
  prints pages i, 1, 2
\item[\outputonly{0,1,2}]
  prints pages 1, 2
\item[\outputonly{1,3-6}]
  prints pages i, iii, iv, v, vi
\item[\outputonly{4,3-6}]
  prints pages iv, 3, 4, 5, 6
\item[\outputonly{5,2}]
  prints pages v, 2
\item[\outputonly{5-2}]
  prints only page v    only!
\item[\outputonly{5-7}]
  prints pages v, vi     only!
\item[\outputonly{1-33}]
  prints pages i, ii, iii, iv, v, vi  only!
\item[\outputonly{29-99}]
  prints pages 29, 30, 31, 32, 33
\item[\outputonly{19-99}]
  prints pages 19, 20, 21, 29, 30, 31, 32, 33
\item[\outputonly{24-32}]
  prints pages 29, 30, 31, 32
\item[\outputonly{15-25,32}]
  prints pages 15, 16, 17, 18, 19, 20, 21, 32
\item[\outputonly{0,4,23-27,31-99}]
  prints pages 4, 31, 32, 33
\end{labeling}

Version 0.9 (test), Sept 25, 1992:
Send problem reports to \mail{asnd@triumf.ca}

\end{document}
