\documentclass[parskip=half, pagesize=auto, version=last]{scrartcl}

\usepackage{fixltx2e}
\usepackage{lmodern}
\usepackage[T1]{fontenc}
\usepackage{textcomp}
\usepackage{microtype}
\usepackage{hyperref}

\addtokomafont{title}{\rmfamily}

\title{The \textsf{bracketkey} package\thanks{This manual corresponds to \textsf{bracketkey}~v1.0, dated~2009/09/24.}}
\subtitle{A \LaTeX\ class for producing bracketed identication keys}
\author{Christoph Heibl}
\date{2009/09/24}


\begin{document}

\maketitle

\noindent
This work may be distributed and/or modified under the
conditions of the \LaTeX\ Project Public License, either version~1.3
of this license or (at your option) any later version.
The latest version of this license is in
\url{http://www.latex-project.org/lppl.txt}
and version~1.3 or later is part of all distributions of \LaTeX\ %
version~2005/12/01 or later.

The use of \textsf{bracketkey} is explained below. See also the example in 
\texttt{Malva.tex} for how to use \textsf{bracketkey}. 
%
\begin{enumerate}
\item Use \verb+\begin{key}{<text>}+ and \verb+\end{key}+ to define the 
  \texttt{bracketkey} environment.

\item The second argument of \verb+\begin key+ can be used to pass 
  an abbreveated (genus) name to the `\texttt{name}' command.

\item The two alternatives of each couplet are defined by:
  
  \verb+\leadONE{<backref>}{<text>}+~~and \\
  \verb+\leadTWO{<text>}+.
  
  \verb+<backref>+ is an integer which refers to the previous couplet 
  number, in case that the current and the previouos couplet
  numbers are not consecutive.

\item Each `\texttt{lead}' command is followed by a `\texttt{name}' command:
  
  \verb+\name{<text>}{<text|integer>}{<text>}+
  
  \verb+<text>+ in the first argument is set in boldface. It can be used 
  for taxon ranks which do not require italized writing.

  \verb+<text>+ in the second argument is set in italics. It typically 
  takes names of genus rank and below. Alternatively, \verb+<integer>+
  refers to the next couplet.
  
  \verb+<text>+ in the third argument is set ``as is''. It can be used to provide 
  authority names.
\end{enumerate}

Please email commentaries, questions, etc. to \texttt{heibl at lmu.de}.

\end{document}
