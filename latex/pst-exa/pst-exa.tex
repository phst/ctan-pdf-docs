\documentclass[pagesize=auto, fontsize=14pt, DIV=9, parskip=half]{scrartcl}

\usepackage{fixltx2e}
\usepackage{etex}
\usepackage{lmodern}
\usepackage[T1]{fontenc}
\usepackage{textcomp}
\usepackage[utf8]{inputenc}
\usepackage{microtype}
\usepackage{hyperref}

\newcommand*{\mail}[1]{\href{mailto:#1}{\texttt{#1}}}
\newcommand*{\pkg}[1]{\textsf{#1}}
\newcommand*{\env}[1]{\texttt{#1}}

\addtokomafont{title}{\rmfamily}

\title{The \pkg{pst-exa} package\thanks{This manual corresponds to \pkg{pst-exa.sty}~v0.01, dated~2010/01/01.}}
\author{Herbert Voß\thanks{\mail{hvoss@tug.org}}}
\date{2010/01/01}


\begin{document}

\maketitle

The package \pkg{pst-exa} was created to realize examples with printed code and output
side by side or on top of each other. The package looks in the image directory for the source
code of the examples and inserts only the code between the environment \env{document},
which is the sequence \verb+\begin{document} ... \end{document}+.

The package provides the environment \env{PSTexample} with the optional
arguments. For more information read the documentation of \texttt{pst2pdf}.

Using \pkg{pst-exa} makes only sense together with the Perl script \texttt{pst2pdf}, which
allows the use of Postscript code and running the document in \textsc{pdf} mode.

\end{document}
