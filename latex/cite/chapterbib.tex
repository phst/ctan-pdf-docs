\documentclass[DIV=9, pagesize=auto]{scrartcl}

\usepackage{etex}
\usepackage{fixltx2e}
\usepackage{xspace}
\usepackage{lmodern}
\usepackage[T1]{fontenc}
\usepackage{textcomp}
\usepackage{xcolor}
\usepackage{booktabs}
\usepackage{listings}
\usepackage{microtype}
\usepackage[unicode=true]{hyperref}

\newcommand*{\mail}[1]{\href{mailto:#1}{\texttt{#1}}}
\newcommand*{\pkg}[1]{\textsf{#1}}
\newcommand*{\cs}[1]{\texttt{\textbackslash#1}}
\makeatletter
\newcommand*{\cmd}[1]{\cs{\expandafter\@gobble\string#1}}
\makeatother
\newcommand*{\env}[1]{\texttt{#1}}
\newcommand*{\meta}[1]{\textlangle\textsl{#1}\textrangle}
\newcommand*{\marg}[1]{\texttt{\{}\meta{#1}\texttt{\}}}
\newcommand*{\oarg}[1]{\texttt{[}\meta{#1}\texttt{]}}
\newcommand*{\BibTeX}{Bib\kern-0.08em\TeX\@\xspace}
\newcommand*{\BigTeX}{Big\kern-0.08em\TeX\@\xspace}

\pdfstringdefDisableCommands{%
  \def\BibTeX{BibTeX\xspace}%
  \def\BigTeX{BigTeX\space}%
}

\addtokomafont{title}{\rmfamily}

\lstset{%
  columns=fullflexible,%
  basicstyle=\ttfamily\footnotesize,
  commentstyle=\color{darkgray},%
  language=[LaTeX]TeX%
}

\title{The \textsf{chapterbib} package\thanks{This manual corresponds to \textsf{chapterbib}~v1.14, dated~31--Mar--2009.}}
\author{Donald Arseneau}
\date{31--Mar--2009}


\begin{document}

\maketitle

\tableofcontents

\overfullrule=10pt



\section{List of Changes}

\begin{tabular}{@{}lcl@{}}
  \toprule
  \multicolumn{2}{c}{\textbf{Version}}                                      \\
  \midrule
  1.14 & (31--MAR--2009) & DA (\verb+\nocite*+, \cmd{\CitationPrefix}, etc) \\
  1.13 & (11--DEC--2007) & DA (new \texttt{final-bib} titling)              \\
  1.12 & (21--MAR--2004) & DA                                               \\
  1.11 & (29--FEB--2004) & DA (allow \cmd{\nocite} check)                   \\
  1.10 & (23--JUN--2003) & DA (\cmd{\bibsection} \cmd{\CBMainSectioning})   \\
  1.9  & (19--SEP--2001) & DA (\texttt{sectionbib} change; \pkg{babel})     \\
  1.8  & (29--APR--1999) & DA (\texttt{gather}, \texttt{duplicate}, toc)    \\
  1.7  & (21--JUL--1997) & DA (\texttt{sectionbib}, \cmd{\nocite})          \\
  1.6  & (08--FEB--1997) & Donald Arseneau (more \texttt{sectionbib})       \\
  1.5  & (09--OCT--1995) & Donald Arseneau (\texttt{rootbib})               \\
  1.4  & (11--MAR--1995) & Donald Arseneau (\texttt{sectionbib})            \\
  1.3  & (04--JUL--1994) & Donald Arseneau (2e, \texttt{cbunit})            \\
  1.2  & (21--MAY--1993) & Donald Arseneau (bug fix)                        \\
  1.1  & (24--MAR--1993) & Donald Arseneau                                  \\
  1.0  & (23--NOV--1988) & Niel Kempson                                     \\
  \bottomrule
\end{tabular}



\section{User Guide}

Allow multiple bibliographies in a \LaTeX\ document, including items
\cmd{\cite}'d in more than one bibliography.  Despite the name ``\pkg{chapterbib}'',
the BIBLIOGRAPHIES ARE FOR EACH INCLUDED FILE, not necessarily for each
chapter.  The main point is to allow you to use \BibTeX: Each included
file should have its own \cmd{\bibliographystyle} and \cmd{\bibliography} commands,
and you should run \BibTeX on each included file separately rather than
on the root file.

There are also the commands \verb+\begin{cbunit}+, \verb+\end{cbunit}+, and \cmd{\cbinput}
to allow multiple bibliographies without using \cmd{\include} (see item~\ref{item:4}).
There are two added hooks, \cmd{\citeform} and \cmd{\citepunct}, which you can 
redefine to customise the formatting of each entry in a citation list,
and the declaration \cmd{\CitationPrefix} to use in preference to \cmd{\citeform}.


\subsection{Usage, Restrictions, and Options}

\begin{enumerate}
\item \label{item:1} Normal use: Put \cmd{\bibliographystyle} and \cmd{\bibliography} commands in
  each \cmd{\include}'d file. Run \LaTeX; run \BibTeX on each included file;
  run \LaTeX; run \LaTeX.

\item The \cmd{\bibliography} and \cmd{\bibliographystyle} commands are not normally used
  in the root file, only in files that have been \cmd{\include}'d. To have a
  whole-document bibliography, see items \ref{item:6}--\ref{item:9}, depending on which style of
  whole-document bib.

\item If you can't use \cmd{\include} because a new section must start below the
  preceding bibliography on the same page (odd format!), then you can
  use \verb+\begin{cbunit}+\allowbreak\texttt{...}\verb+\end{cbunit}+ or \cmd{\cbinput}, with a \verb+{thebibliography}+
  environment in each unit or input file.  To use \BibTeX: input separate
  files using \cmd{\cbinput}; at first use the package or global option \texttt{[draft]},
  run \LaTeX\ on the document, then \BibTeX on each file that was \cmd{\cbinput};
  finally, remove the \texttt{[draft]} option and run \LaTeX\ again (maybe twice to
  get page references right).  The \texttt{[draft]} option only affects the treatment
  of \cmd{\cbinput}, not \cmd{\include} or \verb+\begin{cbunit}+.
  
  With old \LaTeX, do the preliminary run using \cmd{\include} commands, and
  change these to \cmd{\cbinput} for the final run(s).

\item \label{item:4} Your preferred citation style (\texttt{xxx.sty}) may not work with \pkg{chapterbib} at
  first, but it is easy to make it compatible:  In `\texttt{xxx.sty}' change every
  ``\verb+@\@citeb+'' to ``\verb+@\@citeb\@extra@b@citeb+'', and insert the line
  % 
\begin{lstlisting}
\@ifundefined{@extra@b@citeb}{\def\@extra@b@citeb{}}{}
\end{lstlisting}
  % 
  somewhere (but not as a comment or as part of another definition!).
  If the package also redefines \cmd{\bibcite} then you should change that
  definition, replacing ``\verb+@#1+'' with ``\verb+@#1\@extra@binfo+'', and insert
  % 
\begin{lstlisting}
\gdef\@extra@binfo{}
\end{lstlisting}
  % 
  somewhere in the file.
  
  Some citation packages deviate quite far from \LaTeX's own method of
  organizing cite tags using ``\verb+b@\@citeb+''.  The instructions above catch 
  such extensions as ``\verb+Y@\@citeb+'', but not more radical differences.
  In such cases, try contacting the author of the citation package.
  If a citation style does not define ``\cmd{\nocite}'', then that command
  would not be converted when you make the patches to ``\verb+@\@citeb+''.
  Chapterbib will try to detect the hook in ``\cmd{\nocite}'', but if this fails
  you may need to redefine ``\cmd{\nocite}'' (with any ``\verb+@\@citeb+'' changed to 
  ``\verb+@\@citeb\@extra@b@citeb+'') in that sty file.

\item The \pkg{report} and \pkg{book} document classes usually treat the bibliography as
  an unnumbered chapter (\verb+\chapter*+), which is not so good for bibliographies
  IN a chapter.  You can specify
  % 
\begin{lstlisting}
\usepackage[sectionbib]{chapterbib}
\end{lstlisting}
  % 
  to convert your bibliographies from \verb+\chapter*+ to \verb+\section*+, with an
  entry in the table of contents and the page-header.  A bibliography in
  the root file remains as a \verb+\chapter*+.  The \texttt{[sectionbib]} option modifies
  the existing thebibliography environment (or the \cmd{\bibsection} command, if
  present already), so the other formatting in the  bibliography should
  remain unchanged.  On the other hand, if you already have a non-standard
  bibliography defined, or if you want them numbered, it may be easier to
  redefine \cmd{\thebibliography} directly, without any trickery.
  
  Alternatively, you can use the \cmd{\sectionbib} command directly in the
  document preamble.  It takes two parameters: the sectioning command, and
  the name of the sectioning level.  For instance, the \texttt{[sectionbib]} option
  does \verb+\sectionbib{\section*}{section}+. Again, for the most control, it is
  better to redefine \cmd{\thebibliography} entirely.

\item \label{item:6} If you want a completely unrelated bibliography in the root file, perhaps
  for a general reading list, you can provide your own bibliography there
  using the \env{thebibliography} environment.  I don't suppose this will appeal
  to \BibTeX users!

\item To have a cohesive bibliography for the whole document, plus individual
  bibs in the chapters, put \cmd{\bibliography} commands in the included chapters
  plus in the root file.  Make sure the \cmd{\bibliographystyle} for the overall
  bibliography appears FIRST, before any chapters are included. Run \LaTeX;
  run \BibTeX on the root file; run \BibTeX on each included file; run \LaTeX; 
  run \LaTeX. This produces an independent `overall' bibliography which only
  makes sense for various `named' bibliography styles; a numbered style, or
  one with any type of automatic enumeration (like \textsf{Me2007a}, \textsf{Me2007b}) will 
  give unrelated numbers in each bibliography and lead to confusion.

  \begin{sloppypar}
    \BibTeX will complain about multiple \cmd{\bibdata} commands when it makes
    the whole bibliography, but it should obey the first.  If you don't 
    want to see any error messages from bibtex, or if you don't want to put 
    the main \cmd{\bibliographystyle} command first in the document, then use 
    \verb+\usepackage[rootbib]{chapterbib}+ when you run \LaTeX\ first; run \BibTeX 
    on the root file; change to \verb+\usepackage{chapterbib}+; run \LaTeX; run 
    \BibTeX on each included file; run \LaTeX; run \LaTeX.
  \end{sloppypar}

\item \label{item:8} To have a bibliography-by-chapter at the end instead of separate bibs
  in the chapters, use \verb+\usepackage[gather]{chapterbib}+, put \cmd{\bibliography}
  commands in each file, and at the end of the main file. Run \LaTeX\ as
  in item~\ref{item:1}. You can control the titling of the final bibliographies by
  defining \cmd{\FinalBibTitles}, such as
  % 
\begin{lstlisting}
\newcommand\FinalBibTitles{References for Chapter \thechapter}
\end{lstlisting}
  A similar effect may be achived by RE-defining \cmd{\FinalBibPrefix} as
  % 
\begin{lstlisting}
\renewcommand\FinalBibPrefix{References for }
\end{lstlisting}
  % 
  Even more control is achieved by redefining \cmd{\StartFinalBibs}.
  The default definition is (like)
  % 
\begin{lstlisting}
\newcommand{\StartFinalBibs}{%
  \renewcommand{\bibname}{Bibliography for chapter n}}
\end{lstlisting}
  % 
  normally, but when using the \texttt{[sectionbib]} option it becomes
  % 
\begin{lstlisting}
\newcommand{\StartFinalBibs}{\chapter*{\bibname}%
  \addcontentsline{toc}{chapter}{\bibname}\@mkboth{\bibname}{\bibname}%
  \renewcommand{\bibname}{Chapter n}}
\end{lstlisting}
  % 
  where the \cmd{\bibname} text is now provided by \verb+\@auto@bibname+, which 
  relies on bookkeeping and \cmd{\FinalBibPrefix}. 

  If your document class has neither section nor chapter, then you must
  define \cmd{\StartFinalBibs} and also indicate the sectioning: for example,
  if the main sectioning command in your document class is \cmd{\motif}:
  % 
\begin{lstlisting}
\newcommand\CBMainSectioning{motif}
\end{lstlisting}

\item \label{item:9} To have bibliographies in each chapter PLUS a bibliography-by-chapter at
  the end, follow item~\ref{item:9}, but declare \verb+\usepackage[duplicate]{chapterbib}+
  (or \verb+\usepackage[duplicate,sectionbib]{chapterbib}+).

\item If you use Babel, load \pkg{chapterbib} before \pkg{babel}.
\end{enumerate}


\subsection{\cmd{\citeform}, \cmd{\citepunct} and \cmd{\CitationPrefix}:}

Normally, the citations are formatted as given, but you can define \cmd{\citeform}
(with one parameter) to reformat every citation.  Some possibilities:
%
\begin{lstlisting}
\renewcommand\citeform[1]{\romannumeral 0#1}% roman numerals:  [iv,x]
\renewcommand\citeform[1]{(#1)}             % parentheses:  [(3),(4),(7)]
\end{lstlisting}
%
If you change \cmd{\citeform}, you should probably define \verb+\@biblabel+ to match.

Another not-so-good example to provide a chapter-number prefix is
%
\begin{lstlisting}
\renewcommand\citeform[1]{\thechapter.#1}   % number by chapter.
\end{lstlisting}
%
This partially works, but has only limited applicability: it does not 
work with cites in the front-matter (TOC, LOF) or with \pkg{hyperref}.
Instead, there is a \cmd{\CitationPrefix} command to apply a prefix to the 
citation numbers (or names) in the bibliographies and \cmd{\cite} commands 
for the included files. Use it by declaring something like
%
\begin{lstlisting}
\CitationPrefix{\thechapter.}
\end{lstlisting}
%
in the preamble.  The prefix will be applied to all the chapter-bibs
but will not be used in an overall (root) bibliography, if you have one.

\cmd{\citepunct} gives the punctuation (comma-penalty-space) between items
in the \cmd{\cite} list.



\section{Implementation notes}

\LaTeX\ normally uses command names in the form \cmd{\b@\meta{TAG}} to associate a mnemonic
\meta{TAG} with the citation name or number.  \pkg{Chapterbib} changes this to a command
like \cmd{\b@\meta{TAG}@-\meta{nn}} where \meta{nn} is a number identifying which included file is in
effect.


\subsection{Tags indicating the citations and/or the input files:}

\begin{labeling}[\hspace{\labelsep}=]{\cmd{\@extra@b@citeb}}
  \item[\cmd{\c@inputfile}]    counter counting included files
  \item[\cmd{\the@ipfilectr}]  (empty) when typesetting from the root file,
  \item                        \verb+@-\the\c@inputfile+ when typesetting from an included file
  \item[\cmd{\@extra@b@citeb}] \verb+\the@ipfilectr+ (just an alias)
  \item[\cmd{\@extra@binfo}]   the value of \verb+@extra@b@citeb+ as saved in \texttt{.aux} files
\end{labeling}


\subsection{\cmd{\b@\meta{FOO}}:}

In the root file, the citation number (or name) is given by \cmd{\b@\meta{FOO}}
just like regular \LaTeX, but in an \cmd{\include}'d file it is given by
\cmd{\b@\meta{FOO}@-\meta{number}}.  Any definition of \cmd{\cite} (\cmd{\@citex}) should refer
to this using\\\verb+\csname b@\@citeb\@extra@b@citeb \endcsname+.


\subsection{\cmd{\include} \cmd{\cbinput} and \cmd{\cbunit}:}

\begin{sloppypar}
  Redefine the \cmd{\include} macro so that when a file is \cmd{\include}'d \ldots\ %
  Increment the file number, (globally) update \cmd{\the@ipfilectr} to be
  \texttt{@-\meta{number}}, and write \verb+\gdef\@extra@binfo{@-+\texttt{\meta{number}}\verb+}+ in the (main) \texttt{.aux}
  file, and do regular \cmd{\include}. When the \cmd{\include}'d file is finished,
  write \verb+\gdef\@extra@binfo{}+ in the (main) \texttt{.aux} file. Globally reset
  \cmd{\the@ipfilectr} to \verb+{}+.  Provide similar \cmd{\cbunit} and \cmd{\cbinput}.
\end{sloppypar}


\subsection{\cmd{\cite}:}

Redefine \cmd{\cite} (\cmd{\@citex} actually) and \cmd{\bibcite} to use the file number
tag along with the specified tag.  \cmd{\@citex} also uses \cmd{\citeform} as a hook
to reformat each individual citation.
Only do the redefinitions if no citation style that supports chapterbib
has been loaded--as indicated by existence of filename tags.


\subsection{\cmd{\nocite}:}

A while ago \LaTeX\ changed its \cmd{\nocite} so it checks the validity of the
(no)citation tags.  This necessitated adding the ``\cmd{\@extra@b@citeb}'' hook
to \cmd{\nocite}.  At first, I only redefined \cmd{\nocite} when redefining \cmd{\cite},
but there existed citation packages that supported \pkg{chapterbib} by putting
\cmd{\@extra@b@citeb} into \cmd{\cite} but which did not define \cmd{\nocite} at all!
These then gave warning messages for every \cmd{\nocite}.  With version~1.7,
I now try exercising the \cmd{\nocite} command with some functions disabled:
(1) \texttt{.aux} file output is locally switched off; (2) The \cmd{\@ifundefined}
command is hacked to merely typeset the text of the tag; (3) this is
done in a disappearing \cmd{\hbox} so it generates no output.  This detects
(by setting a global flag \cmd{\@gtempa}) a \cmd{\nocite} command that executes
\cmd{\@ifundefined} but not \cmd{\@extra@b@citeb}.  In such cases, \cmd{\nocite} must
be redefined.


\subsection{\cmd{\citepunct} and \cmd{\citeform}:}

These customization hooks are present in \texttt{cite.sty}; others (\cmd{\citemid},
\cmd{\citeleft}, \cmd{\citeright}) are not defined here because \cmd{\@cite} is not redefined.
\LaTeX\ has now adopted \cmd{\@cite@ofmt}, which is normally \cmd{\hbox}, so use both.

We redefine \cmd{\@testdef} so that it properly checks whether the
\cmd{\cite} labels have changed.


\subsection{Table of contents etc:}

As of version~1.8, definitions of \cmd{\@extra@b@citeb} are written to the
\texttt{toc}, \texttt{lot}, and \texttt{lof} files to avoid undefined citation tags in those tables.
This is actually a problem in regular \LaTeX with \BibTeX -- cites that
appear in those tables get numbered starting with~1, but they should
be sequenced by where they originated -- but the problem was worse with
\pkg{chapterbib}.


\subsection{\cmd{\CitationPrefix}:}

This is a really ugly hack, but \cmd{\citeform} worked poorly for putting
a chapter-identifier on each cite.  It locally redefines both
\cmd{\@bibitem} and \cmd{\@lbibitem} at each \cmd{\bibliography}.  The insertion
of the prefix is straightforward for \cmd{\@lbibitem}, but for \cmd{\@bibitem}
the prefix is inserted by a redefined \cmd{\value} macro (eeek!). 



\section{Authors}

\begin{tabular}{@{}ll@{}}
  Niel Kempson                                                             & (original)             \\
  Donald Arseneau  \href{mailto:asnd@triumf.ca}{\texttt{<asnd@triumf.ca>}} & March 1993 -- Mar 2009 \\
\end{tabular}



\section{Detailed List of Changes}

\sloppy
\begin{labeling}{Changes with ver 1.14:}
\item[Changes with ver 1.14:] No warning from \cmd{\nocite*} in chapters, \cmd{\CitationPrefix} 
\item[Changes with ver 1.13:] Rewrite final-bibs titling
\item[Changes with ver 1.12:] Update \cmd{\@citex} \cmd{\bfseries}
\item[Changes with ver 1.11:] Follow \LaTeX's barring \cmd{\nocite} from preamble
\item[Changes with ver 1.10:] \cmd{\CBMainSectioning}; \cmd{\bibsection} hook
\item[Changes with ver 1.9:] \texttt{sectionbib} remove \cmd{\@mkboth}
\item[Changes with ver 1.8:] options \texttt{gather} and \texttt{duplicate}; \pkg{babel} workarounds; toc
\item[Changes with ver 1.7:] change the \cmd{\sectionbib} command re. headings.\\
  update \cmd{\nocite} and alter \cmd{\citeform}
\item[Changes with ver 1.6:] change the \cmd{\sectionbib} command to take arguments.
\item[Changes with ver 1.5:] rootbib option; item~\ref{item:8} in instructions.
\item[Changes with ver 1.4:] \texttt{sectionbib} and \texttt{draft} options
\item[Changes with ver 1.3:] rewrite:
  The filename tag is defined in each \texttt{.aux} file by \cmd{\include}, not as an
  extra parameter to \cmd{\bibcite}.  \cmd{\@bibitem} \& \cmd{\@lbibitem} are left alone.  Add
  \cmd{\citeform} and \cmd{\citepunct}. Add \cmd{\cbunit} and \cmd{\cbinput} for use without \cmd{\include}.
\end{labeling}
\fussy


\end{document}
