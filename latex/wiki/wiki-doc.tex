\documentclass[pagesize=auto]{scrartcl}

\usepackage{fixltx2e}
\usepackage{etex}
\usepackage{lmodern}
\usepackage[T1]{fontenc}
\usepackage{textcomp}
\usepackage[utf8]{inputenc}
\usepackage{microtype}
\usepackage{hyperref}

\newcommand*{\pkg}[1]{\textsf{#1}}
\newcommand*{\cs}[1]{\texttt{\textbackslash#1}}
\makeatletter
\newcommand*{\cmd}[1]{\cs{\expandafter\@gobble\string#1}}
\makeatother
\newcommand*{\env}[1]{\texttt{#1}}
\newcommand*{\opt}[1]{\texttt{#1}}
\newcommand*{\meta}[1]{\textlangle\textsl{#1}\textrangle}
\newcommand*{\oarg}[1]{\texttt{[}\meta{#1}\texttt{]}}

\addtokomafont{title}{\rmfamily}

\title{The \pkg{wiki} package\thanks{This manual corresponds to \pkg{wiki.sty}~v0.2, dated~2008/07/02.}}
\author{Uwe Lück\thanks{\url{http://contact-ednotes.sty.de.vu}}}
\date{2008/07/02}


\begin{document}

\maketitle

\section{Features}

The `\pkg{wiki}' package enables \LaTeX\ users to use some of the
markup used in editing Wikipedia pages, cf.
%
\begin{quote}
  \url{http://en.wikipedia.org/wiki/Wikipedia:Cheatsheet}
\end{quote}
%
or German:
%
\begin{quote}
  \url{http://de.wikipedia.org/wiki/Hilfe:Bearbeitungshilfe}\\
  \url{http://de.wikipedia.org/wiki/Hilfe:Textgestaltung}
\end{quote}

Advantages of such a kind of markup are:
%
\begin{enumerate}
\item easier and faster to type than \LaTeX\ code;

\item intuitive, understandable at a glance;

\item in reading the source code of the document,
  the markup doesn't much distract from the \emph{text}
  to be printed.
\end{enumerate}

There is a binary `\texttt{easylatex}' available on CTAN which is more
powerful than the `\pkg{wiki}' package; the idea here is that a \emph{macro
package} doing something similar is easier to use and to customize.


\subsection{What the Package Supports}

\begin{enumerate}
\item \texttt{\textit{\textquotesingle\textquotesingle Italics\textquotesingle\textquotesingle}} are marked up by surrounding pairs of apostrophes,
  \texttt{\textbf{\textquotesingle\textquotesingle\textquotesingle boldface\textquotesingle\textquotesingle\textquotesingle}} needs one apostrophe more on each side.
  Boldface and italics may be nested or overlap.
  (This endangers quoting, but I have dealt with it.)

\item Surrounding ``equals'' symbols (`\texttt{=}') mark up section headings
  etc.: `\verb+== Section ==+', `\verb+=== Subsection ===+', down to
  `\verb+===== Subparagraph =====+'.

\item If a line starts with `\texttt{*}', it becomes an item in an
  `\env{itemize}' environment. Likewise, `\texttt{\#}' creates an `\env{enumerate}'
  environment, and `\texttt{;}' makes an item in a `\env{discription}'
  environment (differs from Wiki! -- since I don't understand
  what the Wiki version is good for). `\texttt{:}' makes a `\env{quote}' --
  while on Wikipedia instead is used for comments on talk
  pages. But it is also used for indented math displays,
  so it may be combined with \verb+$...$+ here.
  An indented line is typeset \env{verbatim}. Each of these
  environments is ended by a code line without leading blank
  spaces.
\end{enumerate}


\subsection{Limitations}

\begin{enumerate}
\item The markup ``language'' provided here replaces a few most
  common \LaTeX\ commands only in their most simple versions.
  (Yet I think that already this makes the code more readable;
  the more special cases are very rare.) E.\,g.,\ it's not
  possible to suppress the italic correction as with
  `\verb+\textit{...\nocorr}+'. However, in ``normal'' cases it
  should be possible to use the more powerful \LaTeX\ markup
  mixed with ``Wiki'' markup.

\item The parsing algorithms employed here differ in outcome to
  those of Wikipedia in difficult cases.

\item Tables, nesting lists, links, Wiki templates and tags
  are not supported -- and I don't expect that I can afford
  improving this soon!

\item I still have encountered problems that I didn't understand\dots
\end{enumerate}


\section{Package Options and User Commands}

\sloppy
\cmd{\usepackage}\oarg{options}\verb+{wiki}+, of course. There are \meta{options}
`\opt{noEnvironments}', `\opt{noFonts}', `\opt{noHeadings}' giving up
functionality for lists etc.,\ for italics and boldface,
for section headings etc.,\ respectively.

There are commands \cmd{\wikimarkup}, \cmd{\wikiEnvironments},
\cmd{\wikiFonts}, \cmd{\wikiHead\-ings} to turn respective functionality on
(unless disabled by package option). \cmd{\nowiki\-markup},
\cmd{\nowikiEnvironments}, \cmd{\nowikiFonts}, \cmd{\nowikiHeadings} turn them
off.

\fussy
\textbf{Vital:} The package does \emph{not} execute \cmd{\wikimarkup}
because this would have strange effects, e.\,g.,\ packages
loaded later may break. The user must put activating
commands into the `\env{document}' environment, maybe just preceding
the first heading.

Dangers to be avoided by \cmd{\nowiki\dots} are:
%
\begin{itemize}
\item \cmd{\wikiHeadings} (which also is executed by \cmd{\wikimarkup})
  disables \TeX\ assignments using `\texttt{=}'.

\item \cmd{\wikiEnvironments} (executed by \cmd{\wikimarkup}) executes macros
  at every end of a code line (containing no comment marke),
  and it tries to typeset a following line verbatim if it is
  indented.
\end{itemize}

\end{document}
