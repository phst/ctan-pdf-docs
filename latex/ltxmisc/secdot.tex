\documentclass[pagesize=auto, fontsize=12pt, DIV=10, parskip=half]{scrartcl}

\usepackage{fixltx2e}
\usepackage{etex}
\usepackage{lmodern}
\usepackage[T1]{fontenc}
\usepackage{textcomp}
\usepackage[svgnames]{xcolor}
\usepackage{listings}
\usepackage{microtype}
\usepackage{hyperref}

\newcommand*{\mail}[1]{\href{mailto:#1}{\texttt{#1}}}
\newcommand*{\pkg}[1]{\textsf{#1}}
\newcommand*{\cls}[1]{\textsf{#1}}
\newcommand*{\cs}[1]{\texttt{\textbackslash#1}}
\makeatletter
\newcommand*{\cmd}[1]{\cs{\expandafter\@gobble\string#1}}
\makeatother
\newcommand*{\meta}[1]{\textlangle\textsl{#1}\textrangle}
\newcommand*{\marg}[1]{\texttt{\{}\meta{#1}\texttt{\}}}

\addtokomafont{title}{\rmfamily}

\lstset{%
  language=[LaTeX]TeX,%
  columns=flexible,%
  upquote=true,%
  numbers=left,%
  basicstyle=\ttfamily,%
  keywordstyle=\color{Navy},%
  commentstyle=\color{DimGray},%
  stringstyle=\color{SeaGreen},%
  numberstyle=\scriptsize\color{SlateGray},%
  escapechar=\$%
}

\title{The \pkg{secdot} package\thanks{This manual corresponds to \pkg{secdot.sty}~v2.0, dated~July 2000.}}
\subtitle{Define section numbers with dots}
\author{%
  Robin Fairbairns\thanks{\mail{rf10@cl.cam.ac.uk}}%
  \and Steve Grathwohl\thanks{Duke University}%
}
\date{July 2000}


\begin{document}

\maketitle

the package as loaded causes section numbers to be output with a dot
after them.

the command \cmd{\sectiondot}\marg{level} will make `\meta{level}' sections also be
output with a dot after them -- an example of use would be:
%
\begin{lstlisting}
\sectiondot{subsection}
\end{lstlisting}

the command \cmd{\sectionpunct}\marg{level}\marg{punctuation} gives finer
control.  examples of use would be
%
\begin{lstlisting}
\sectionpunct{section}{. } % \sectiondot places a \quad after the
                              % dot, which may look excessive
\sectionpunct{section}{\quad} % restores default latex behaviour
\end{lstlisting}

other eccentricities of numbering could be coded by those with
stronger stomachs, by defining the appropriate
%
\begin{lstlisting}
\csname @seccntfmt@$\meta{level}$\endcsname
\end{lstlisting}
%
which macro will take one argument, when invoked: the `\meta{level}' name

\end{document}
