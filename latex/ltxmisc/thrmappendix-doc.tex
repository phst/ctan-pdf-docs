\documentclass[pagesize=auto]{scrartcl}

\usepackage{fixltx2e}
\usepackage{etex}
\usepackage{xspace}
\usepackage{lmodern}
\usepackage[T1]{fontenc}
\usepackage{textcomp}
\usepackage[svgnames]{xcolor}
\usepackage{amsmath}
\usepackage{listings}
\usepackage{microtype}
\usepackage{hyperref}

\newcommand*{\secbreak}{\par\medskip\centerline{$*\quad*\quad*$}\par\medskip}
\newcommand*{\mail}[1]{\href{mailto:#1}{\texttt{#1}}}
\newcommand*{\pkg}[1]{\textsf{#1}}
\newcommand*{\cs}[1]{\texttt{\textbackslash#1}}
\makeatletter
\newcommand*{\cmd}[1]{\cs{\expandafter\@gobble\string#1}}
\makeatother
\newcommand*{\meta}[1]{\textlangle\textsl{#1}\textrangle}
\newcommand*{\marg}[1]{\texttt{\{}\meta{#1}\texttt{\}}}

\addtokomafont{title}{\rmfamily}

\lstset{%
  language=[LaTeX]TeX,%
  columns=flexible,%
  upquote=true,%
  numbers=left,%
  basicstyle=\ttfamily,%
  keywordstyle=\color{Navy},%
  commentstyle=\color{DimGray},%
  stringstyle=\color{SeaGreen},%
  numberstyle=\scriptsize\color{SlateGray}%
}

\title{The \pkg{thrmappendix} package}
\author{Ben Klemens\thanks{\mail{b@avocado.caltech.edu}}}
\date{2 Jan 2003}


\begin{document}

\maketitle

\begin{quote}
  \footnotesize
  \texttt{thrmappendix.sty} \textcopyright~2003, Ben Klemens

  This program is free software; you can redistribute it and/or modify
  it under the terms of the GNU General Public License as published by
  the Free Software Foundation; either version~2 of the License, or
  (at your option) any later version.
\end{quote}

\noindent
This is my implementation of theorems, lemmas, \&c.  The main goal is
to let you reprint theorems in the appendix with their proofs, so the
text isn't cluttered with long proofs, but the reader doesn't have to
keep flipping back to the text to find out what you're talking about in
the appendix.

It also has the side benefits that the syntax here involves less typing
and fewer redundancies than the default, so I use it even for things
that won't ever be proven. Compare the standard:
%
\begin{lstlisting}
\begin{definition} \label{xdef} 
Let $x$ be a number.
\end{definition}
\end{lstlisting}
%
versus the alternative here:
%
\begin{lstlisting}
\stmt{defn}{xdef}{
Let $x$ be a number.
}
\end{lstlisting}

\secbreak

Here is a quick summary of the macros defined in this file:
%
\begin{description}
\item[\cmd{\stmt}\marg{statement\textunderscore type}\marg{label}\marg{This is what I want to say}]
  The definition of a theorem, assumption, \&c.
\item[\cmd{\stmtproof}\marg{label}\marg{This is the proof.}]
  Where you should write your proof, for use with:
\item[\cmd{\rptstmtwithproof}\marg{label}]
  This states the theorem and proof together.
\item[\cmd{\rptstmt}\marg{label}]
  This restates just the theorem.
\item[\cmd{\refstmt}\marg{label}]
  Prints, e.g., `Theorem 7'---like \cmd{\ref}\marg{label} but you don't have
  to remember if it's a theorem, corollary, or lemma.
\item[\cmd{\comment}\marg{blather}]
  Delineates comments without all those darn percent signs.
\end{description}

\secbreak

Here is a sample document using most all of the above. You can cut and
paste it to another file and compile it to see what it does. 
%
\begin{lstlisting}
\documentclass{article}
\usepackage{thrmappendix}
\begin{document}

\section{A theorem}

\stmt{thrm}{sample}{Socrates is mortal.}

\refstmt{sample} can either be proven using data (the fact that
Socrates is dead), or by the proof which is provided in the appendix.



\section{Appendix}
In this appendix, the reader will find proofs of theorems
not given in the text.

\stmtproof{sample}{Socrates is a man. All men are mortal.}

\rptstmtwithproof{sample}

\comment{Notice that you can put the \stmtproof{sample}{...} anywhere
you want, including right after your statement, just before using
\rptstmtwithproof, or a separate file (then use \input{proofs.tex}).
That way, neither your finished paper nor your source will be
cluttered with proofs.}
\end{document}
\end{lstlisting}

\secbreak

Here are the definitions of the types of theorems you can use
with these commands, and the proof markers:

\begin{lstlisting}
\long\def\theproofmarkersforthrmappendix{

%Format:
%\newtype{nickname}{numbered like}{title}

%inputs
\newtype{ass}{ass}{Assumption}
\newtype{defnition}{ass}{Definition}
\newtype{defn}{ass}{Definition}

%outputs
\newtype{statement}{statement}{Proposition}
\newtype{prop}{statement}{Proposition}
\newtype{thrm}{statement}{Theorem}
\newtype{corr}{statement}{Corollary}
\newtype{lma}{statement}{Lemma}
\newtype{guess}{statement}{Conjecture}


\def\pf{{\bf Proof: }}
\def\endpf{\hfill $\diamondsuit$}
}
\end{lstlisting}
%
These are my preferred environments (and numbering scheme);
add your own or modify to suit. The format here is:
\cmd{\newtype}\marg{nickname}\marg{numbered like}\marg{title}
						
E.g., the second line lets me put
%
\begin{lstlisting}
\stmt{thrm}{mortalityresult}{Socrates is mortal.}
\end{lstlisting}
%
in the \TeX\ file, since it defined the nickname `\texttt{thrm}'. If the statement
counter is up to~\#7, then the output will be:
%
\begin{quote}
  Theorem 7: Socrates is mortal.
\end{quote}
%
since \texttt{thrm}s use the statement counter and print `Theorem'. To refer to
this theorem in the text, you have two choices:
%
\begin{flushleft}
  \begin{tabular}{l@{$\;\implies\;$}l}
    \lstinline|\refstmt{mortalityresult}|     & Theorem 7 \\
    \lstinline|Theorem \ref{mortalityresult}| & Theorem 7
  \end{tabular}
\end{flushleft}
%
The first form is nice because after you prove your result, you can
change it from a theorem to a conjecture without hunting through the
text to fix every reference.

You can see that all of the results use the `\texttt{statement}' counter and
all of the inputs use the `\texttt{ass}' counter.  Proposition~1 may be followed
by Theorem~2, followed by Corollary~3.  This is my preferred numbering
scheme, suggested by Paolo Ghirardato.  After all, the intent of numbering
these things is not to keep score as to how many theorems you've got,
but to make it easy for the reader to find what you're referring to;
a consistent numbering system facilitates this (especially in a long proof
appendix).

[On the other hand, a referee of one of my papers complained that
17 theorems and 18 corollaries was too many, and I could cut that in
half. Of course, there were only 18 results total, and s/he was only
glancing at the two results on the last page.]

To use independent numbering for each type, simply change the second field
in the definitions above to match the first field in each case. E.g.:
%
\begin{lstlisting}
\newthrm{thrm}{thrm}{Theorem}
\end{lstlisting}
%
will allow you to count theorems separately from other results.

The \cmd{\rptstmtwithproof} macro here will include a marker at the beginning
and end of the proof, so you don't have to include them yourself.
The macros \cmd{\pf} and \cmd{\endpf} are defined above and can be redefined to
suit. Popular endproof markers include:
%
\begin{lstlisting}
\def\endpf{$blackbox$}
\def\endpf{$blacklozenge$}
\def\endpf{{\sc QED}}
\end{lstlisting}

\end{document}
