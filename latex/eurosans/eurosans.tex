\documentclass[DIV=9, pagesize=auto]{scrartcl}

\usepackage{fixltx2e}
\usepackage{etex}
\usepackage{xspace}
\usepackage{lmodern}
\usepackage[T1]{fontenc}
\usepackage{textcomp}
\usepackage{array}
\usepackage{booktabs}
\usepackage{microtype}
\usepackage[unicode=true]{hyperref}

\newcommand*{\mail}[1]{\href{mailto:#1}{\texttt{#1}}}
\newcommand*{\pkg}[1]{\textsf{#1}}
\newcommand*{\cls}[1]{\textsf{#1}}
\newcommand*{\cs}[1]{\texttt{\textbackslash#1}}
\makeatletter
\newcommand*{\cmd}[1]{\cs{\expandafter\@gobble\string#1}}
\makeatother
\newcommand*{\opt}[1]{\texttt{#1}}

\addtokomafont{title}{\rmfamily}

\title{The \pkg{eurosans} package\thanks{This manual corresponds to \pkg{eurosans}~v3.1, dated~2004-01-26.}}
\author{Walter Schmidt\\\mail{w.a.schmidt@gmx.net}}
\date{2004--01--26}


\begin{document}

\maketitle

\section{Usage}

The \LaTeX\ package \pkg{eurosans.sty} provides a convenient \LaTeX\ %
interface for the free Adobe~Euro symbol fonts in Type~1
(PostScript) format.  Loading of the package
%
\begin{verbatim}
\usepackage{eurosans}
\end{verbatim}
%
provides a new command
%
\begin{verbatim}
\euro
\end{verbatim}
%
which typesets an Euro symbol.  The weight (medium or
boldface), shape (upright or oblique) and width (regular or
condensed) varies according to the font currently selected.
The symbol blends well with most typefaces, except for
typewriter fonts.  The medium/upright/regular variant meets
the official design by the EC.

The \pkg{Euro} fonts can be scaled, in order to match the other
fonts in the document, by loading the package with the
option \opt{[scaled=\ldots]}.  For instance, to use the \pkg{Euro} fonts at
90\% of their natural size:
%
\begin{verbatim}
\usepackage[scaled=0.9]{eurosans}
\end{verbatim}
%
The \pkg{eurosans} package requires the package \pkg{keyval.sty}, which
is part of the ``\pkg{graphics}'' bundle and should be available
with any decent \LaTeX\ system.


\section{The font family \texttt{eurosans}}

The command \cmd{\euro} switches to a particular font family named
``\texttt{eurosans}'', which is set up with the encoding ``\texttt{U}'' only.

\minisec{NFSS classification:}

\medskip
\begin{tabular}{@{}lll@{}}
  \toprule
  series                    & shapes                               & PostScript \texttt{FontName}                         \\
  \midrule
  \texttt{m}                & \texttt{n}, \texttt{it}, \texttt{sl} & \texttt{EuroSans-Regular}, \texttt{EuroSans-Italic}  \\
  \texttt{b}, \texttt{bx}   & \texttt{n}, \texttt{it}, \texttt{sl} & \texttt{EuroSans-Bold}, \texttt{EuroSans-BoldItalic} \\
  \texttt{mc}               & \texttt{n}, \texttt{it}, \texttt{sl} & \texttt{EuroMono-Regular}, \texttt{EuroMono-Italic}  \\
  \texttt{sbc}, \texttt{bc} & \texttt{n}, \texttt{it}, \texttt{sl} & \texttt{EuroMono-Bold}, \texttt{EuroMono-BoldItalic} \\
  \bottomrule
\end{tabular}

\medskip
\noindent
(The \texttt{EuroMono} typefaces are actually condensed versions of
\texttt{EuroSans}.)


\section{Obtaning and installing the Euro Fonts}

\renewcommand*{\labelenumi}{(\theenumi)}

\begin{enumerate}
\item Install the font metrics (\texttt{.tfm} files), to be found in
  the CTAN directory \texttt{fonts/euro/tfm}.
  
\item Install a font map file for \texttt{dvips}, pdf\TeX\ etc, to be found
  in the CTAN directory \texttt{fonts/euro/dvips}:
  %
  \begin{itemize}
  \item \texttt{zpeu.map} is for use with dvips and pdf\TeX\ on PC or Unix
    platform, it requires renaming of the fonts according to 
    the Karl-Berry scheme.
  \item \texttt{zpeu-origname.map}: dito, but you need not rename the fonts;
  \item \texttt{zpeu-mac.map} is for use on the Mac platform
  \end{itemize}
  
  A font map file and the \texttt{.tfm} files for the \pkg{Euro} fonts may
  already be provided in your \TeX\ system, so that you need not
  install them manually; please, consult its documentation!

\item Due to legal reasons, the actual \texttt{Type1} fonts (\texttt{.pfb} and
  \texttt{.afm} files) are \emph{not} distributed from CTAN\@.  They can be
  obtained for free from Adobe:

  \url{http://www.adobe.com/type/eurofont.html}

  \begin{enumerate}
  \item PC, Unix
  
    You will receive a self-extracting archive ``\texttt{eurofont.exe}'' for
    DOS/Win, which can be unpacked using Info-Zip's ``\texttt{unzip}''
    program, too.  The archive file can also be downloaded
    immediately:

    \url{ftp://ftp.adobe.com/pub/adobe/type/win/all/eurofont.exe}\\
    \url{ftp://ftp-pac.adobe.com/pub/adobe/type/win/all/eurofont.exe}
  \item Mac
  
    The LWFN-Fonts and the Screenfonts for Apple Macintosh
    Computers can be downloaded immediately as:

    \url{ftp://ftp.adobe.com/pub/adobe/type/mac/all/eurofont.sea.hqx}
  
    You need Aladdin's StuffIt Expander to remove the binhex
    encoding of the self-extracting archive.
  \end{enumerate}

\item
  \begin{enumerate}
  \item PC, Unix

    Move the \texttt{.pfb} and \texttt{.afm} files from the archive to a suitable
    directory of your \TeX\ system.  When using the map file
    ``\texttt{zpeu.map}'', you also have to rename the files as follows:

    \begingroup
    \catcode`\_=12
    \begin{tabular}{@{}>{\ttfamily}l@{\enspace$\to$\enspace}>{\ttfamily}l@{}}
      _1______.PFB & zpeurs.pfb  \\
      _1B_____.PFB & zpeubs.pfb  \\
      _1I_____.PFB & zpeuris.pfb \\
      _1BI____.PFB & zpeubis.pfb \\
      _2______.PFB & zpeurt.pfb  \\
      _2B_____.PFB & zpeubt.pfb  \\
      _2I_____.PFB & zpeurit.pfb \\
      _2BI____.PFB & zpeubit.pfb \\
      _3______.PFB & zpeur.pfb   \\
      _3B_____.PFB & zpeub.pfb   \\
      _3I_____.PFB & zpeuri.pfb  \\
      _3BI____.PFB & zpeubi.pfb
    \end{tabular}
    \endgroup

    \ldots and ditto for the \texttt{.AFM} files!

    When using the map file ``\texttt{zpeu-origname.map}'', you need \emph{not}
    rename the files; however, on Unix you should make sure that
    the names are lower case.
  \item Mac

    If you are using the LWFN fonts for Macs just copy them into
    your \texttt{Fonts} folder in the \texttt{System} folder.

    See the manual of your \TeX\ distribution.for how to use the 
    screen fonts in your \textsc{dvi} viewer.

    For Oz\TeX\ make a new config file or add the new fonts to an
    existing config file like this:

\begin{verbatim}
zpeurs  EuroSanReg  "Euro Sans"  nil
\end{verbatim}
    
    and so on for the other fonts.
  \end{enumerate}
\end{enumerate}


\section{Known problems}

With \texttt{dvips} prior to version~5.83, partial font downloading
fails with an error message such as:
%
\begin{verbatim}
File ... ended before all chars have been found
We scan 0 Chars from 226
Last seen token was '/Euro'
\end{verbatim}
%
Partial font downloading can be turned off by calling \texttt{dvips}
with the option \texttt{-j0} or by specifying \texttt{j0} in the \texttt{dvips}
configuration file.

This bug has been fixed with \texttt{dvips} version~5.83.


\section{History}

\minisec{v3.1 2004--01--26}

\begin{itemize}
\item fixed implementation of the option ``\opt{scaled}'';
\item changed \cmd{\euro} macro to use symbol \#128 with the 
  proper glyph name ``\texttt{Euro}'';
\item added installation instructions for Mac
  (credits to Martin Buchmann)
\end{itemize}
    
\minisec{v3.0 2002--01--09}

optional scaling 
  
\minisec{v2.1 2000--06--15}

the \pkg{EuroMono} fonts are used for the condensed series
  
\minisec{v2.0 1999--07--21}

changed the font names acording to ``Karl Berry'' scheme
  
\minisec{v1.0 1999--01--19}

\cmd{\euro} is now robust; 
documentation update
  
\minisec{v0.9 1998--11--09}

first public version

\end{document}
