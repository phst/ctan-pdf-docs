% \iffalse
%% Package `Tabbing' to use with LaTeX 2e
%% Copyright (C) 1996, 1997, 1998, 1999 Jean-Pierre F. Drucbert, all rights reserved
%%
%% You may use and distribute this file freely, provided that
%% you don't pretend that you wrote it.
%%
% It may be distributed and/or modified under the
% conditions of the LaTeX Project Public License, either version 1.1
% of this license or (at your option) any later version.
% The latest version of this license is in
%    http://www.latex-project.org/lppl.txt
% and version 1.1 or later is part of all distributions of LaTeX 
% version 1999/06/01 or later.
%
%<package>\NeedsTeXFormat{LaTeX2e}[1997/06/01]
%<package>\ProvidesPackage{Tabbing}[1997/12/18 v1.0 Tabbing environment (JPFD)]
%
%<*driver>
\documentclass{ltxdoc}
\GetFileInfo{Tabbong.sty}
\def\filedate{1999/08/03}
\def\fileversion{v1.1}
\EnableCrossrefs         
%\DisableCrossrefs   % Say \DisableCrossrefs if index is ready
\RecordChanges                  % Gather update information
\CodelineIndex                  % Index code by line number
\title{The \pkg{Tabbing} package}
\author{Jean-Pierre F. Drucbert\\\texttt{drucbert@onecert.fr}}%
\date{\filedate}
\def\bs{\texttt{\char'134}}
\let\pkg\textsf
\usepackage{Tabbing}
\begin{document}
\maketitle
\DocInput{Tabbing.dtx}
\end{document}
%</driver>
% \fi
%
% \CheckSum{106}
%
% \changes{v1.0}{18 Dec 97}{First officially released version.}
% \changes{v1.1}{03 Aug 99}{Added LPPL License.}
%
% \DoNotIndex{\@Mii,\@Miv,\@cons,\@currlist,\@dblarg,\@dbldeferlist}
% \DoNotIndex{\@dblfloat,\@dottedtocline,\@eha,\@Esphack,\@float}
% \DoNotIndex{\@floatpenalty,\@ifnextchar,\@ifundefined,\@latexerr}
% \DoNotIndex{\@mkboth,\@namedef,\@nameuse,\@parboxrestore,\@spaces}
% \DoNotIndex{\@starttoc,\@tempa,\@tempboxa,\@tempdima,\@warning}
% \DoNotIndex{\addcontentsline,\addtocounter,\advance,\arabic,\bfseries}
% \DoNotIndex{\bgroup,\box,\chapter,\columnwidth,\csname,\def,\dimen,\docdate}
% \DoNotIndex{\edef,\egroup,\else,\endcsname,\endinput,\expandafter,\fi}
% \DoNotIndex{\filedate,\fileversion,\global,\hbadness,\hbox,\hfil,\hrule}
% \DoNotIndex{\hsize,\ht,\if@twocolumn,\ifdim,\iffalse,\ifnum,\iftrue,\ifvbox}
% \DoNotIndex{\ifx,\ignorespaces,\intextsep,\kern,\let,\long,\moveleft,\newbox}
% \DoNotIndex{\newcommand,\newcounter,\newif,\newsavebox,\noexpand,\normalsize}
% \DoNotIndex{\numberline,\PackageError,\PackageWarning,\par,\parindent}
% \DoNotIndex{\penalty,\prevdepth,\protect,\refstepcounter,\relax}
% \DoNotIndex{\renewcommand,\rmfamily,\section,\setbox,\setcounter,\space}
% \DoNotIndex{\textheight,\the,\typeout,\unvbox,\uppercase,\vadjust,\value}
% \DoNotIndex{\vbox,\vrule,\vskip,\vspace,\wd,\z@}
%
% \begin{abstract}
% This package\footnote{%
% \begin{tabular}[t]{l}
% Copyright \copyright\ 1996, 1997, 1998 by\\
% Jean-Pierre F. Drucbert\vphantom{bp}\\
% ONERA/Centre de Toulouse SRI\vphantom{bp}\\
% Office National d'\'Etudes et de Recherches A\'erospatiales\vphantom{bp}\\
% Centre de Toulouse\vphantom{bp}\\
% Service R\'eseaux et Informatique\vphantom{bp}\\
% Complexe Scientifique de Rangueil\vphantom{bp}\\
% \\
% 2, Avenue \'Edouard Belin\vphantom{bp}\\
% BP 4025 F-31055 TOULOUSE CEDEX\vphantom{bp}\\
% FRANCE\vphantom{bp}\\
% \vphantom{bp}\\
% Email: \texttt{drucbert@onecert.fr}\vphantom{bp}\\
% \end{tabular}}
% provides a `Tabbing' environment, analog to the \LaTeX\ standard
% `tabbing' environment, but allowing accented letters. No more
% |\a'|, |\a`| and |\a=| needed.
% \end{abstract}
%
%%%%%%%%%%%%%%%%%%%%%%%%%%%%%%%%%%%%%%%%%%%%%%%%%%%%%%%%%%%%%%%%%%%%%%%%
% \section{The \pkg{Tabbing} package}
%%%%%%%%%%%%%%%%%%%%%%%%%%%%%%%%%%%%%%%%%%%%%%%%%%%%%%%%%%%%%%%%%%%%%%%%
%
% \newcommand{\tabrule}[1]{\makebox[0pt]{\raisebox
%  {0pt}[0pt]{\rule{\fboxrule}{#1\baselineskip}}}}
% \LaTeX\ provides the \texttt{tabbing} environment. But it is not
% user-friendly when you must use accented letters, whith the grave,
% acute and macron accents, |\'a|~\'a, |\`a|~\`a and |\=a|~\=a. Even
% when you use a good input encoding on 8~bits. So the \pkg{Tabbing}
% package offers a \texttt{Tabbing} environment, which is a slightly
% modified copy of the standard \texttt{tabbing} environment. In the
% \texttt{Tabbing} environment, the local commands |\>|, |\<|, |\=|,
% |\+|, |\-|, |\`| and |\'| are replaced by |\TAB>|, |\TAB<|, |\TAB=|,
% |\TAB+|, |\TAB-|, |\TAB`| and |\TAB'| respectively. Hence the conversion 
% is rather easy. Acute and grave accents are very often used in french,
% this environment should be useful. In the example of
% Figure~\ref{f+Tabbing}, the vertical rules mark the tab stops (with
% the macro |\tabrule|, which is not part of the package)
%    
% \begin{figure}
% \begin{footnotesize}
% \begin{verbatim}
% \newcommand{\tabrule}[1]{\makebox[0pt]{\raisebox
%  {0pt}[0pt]{\rule{\fboxrule}{#1\baselineskip}}}}
%
% \begin{Tabbing}
% gnomon \TAB= agn\=ostic \TAB=     arma\TAB= dillo     \TAB= gnash \TAB= \kill
%        \TAB>            \TAB> gnu     \TAB> gneisses  \TAB>       \TAB> gnarl
%  \\*
%        \TAB>            \TAB> \'ecole \TAB> \'el\`eve \TAB>    \TAB> examen
%  \\*
%        \TAB>            \TAB> �cole   \TAB> �l�ve     \TAB>    \TAB> examen
%  \\*
%        \TAB>            \TAB> u       \TAB> e         \TAB> g     \TAB>
%  \TAB`
% \end{Tabbing}
% \end{verbatim}
% \end{footnotesize}
%
% \begin{minipage}{\textwidth}
% \begin{Tabbing}
% gnomon \TAB= agn\=ostic \TAB= arma\TAB= dillo \TAB= gnash \TAB= \kill
%        \TAB>            \TAB> gnu \TAB> gneisses \TAB>   \TAB> gnarl \\*
%        \TAB> \TAB> \'ecole \TAB> \'el\`eve \TAB>   \TAB> examen \\*
%        \TAB> \TAB> �cole \TAB> �l�ve \TAB>   \TAB> examen \\*
%        \tabrule{2} \TAB>\tabrule{2} \TAB> \tabrule{2}u   \TAB>
%            \tabrule{2}e
%          \TAB> \tabrule{2}g \TAB>\tabrule{2} \TAB`\tabrule{2}
%  \end{Tabbing}
% \end{minipage}
%
% \caption{A simple \texttt{Tabbing} environment}\label{f+Tabbing}
% \end{figure}
%   
% Note that the markup is more visible than in the \texttt{tabbing}
% environment, and the syntax of accented letters is \emph{the same}
% outside and inside of the new \texttt{Tabbing} environment.
% 
% \StopEventually{\setcounter{IndexColumns}{2}\PrintIndex\PrintChanges}
%
% \clearpage
% \section{Implementation}
%
%    \begin{macrocode}
%<*package>
%    \end{macrocode}
%
% \begin{environment}{Tabbing}
% We just copy the standard \texttt{tabbing} environment, and
% add the local macro |\TAB| who tests its argument.
% An error message has been added.
% \DescribeMacro{\TAB}
%    \begin{macrocode}
\gdef\Tabbing{\lineskip \z@skip
% %     \let\>\@rtab
% %     \let\<\@ltab
% %     \let\=\@settab
% %     \let\+\@tabplus
% %     \let\-\@tabminus
% %     \let\`\@tabrj
% %     \let\'\@tablab
\def\TAB##1{\ifx ##1>\@rtab\else
            \ifx ##1<\@ltab\else
            \ifx ##1=\@settab\else
            \ifx ##1+\@tabplus\else
            \ifx ##1-\@tabminus\else
            \ifx ##1`\@tabrj\else
            \ifx ##1'\@tablab\else
                         \PackageError{Tabbing}%
                         {Bad argument ##1 for Tabbing specification}
            \fi\fi\fi\fi\fi\fi\fi}
     \let\\=\@tabcr
     \global\@hightab\@firsttab
     \global\@nxttabmar\@firsttab
     \dimen\@firsttab\@totalleftmargin
     \global\@tabpush\z@ \global\@rjfieldfalse
     \trivlist \item\relax
     \if@minipage\else\vskip\parskip\fi
     \setbox\@tabfbox\hbox{\rlap{\indent\hskip\@totalleftmargin
       \the\everypar}}\def\@itemfudge{\box\@tabfbox}%
     \@startline\ignorespaces}
\gdef\endTabbing{%
  \@stopline\ifnum\@tabpush >\z@ \@badpoptabs \fi\endtrivlist}
%    \end{macrocode}
% \end{environment}
%
% \Finale
% \end{document}
\endinput
