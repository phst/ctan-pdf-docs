\documentclass[DIV=9, parskip=half, pagesize=auto]{scrartcl}

\usepackage{fixltx2e}
\usepackage{etex}
\usepackage{xspace}
\usepackage{lmodern}
\usepackage[T1]{fontenc}
\usepackage{textcomp}
\usepackage{microtype}
\usepackage[unicode=true]{hyperref}

\newcommand*{\mail}[1]{\href{mailto:#1}{\texttt{#1}}}
\newcommand*{\pkg}[1]{\textsf{#1}}
\newcommand*{\cs}[1]{\texttt{\textbackslash#1}}
\makeatletter
\newcommand*{\cmd}[1]{\cs{\expandafter\@gobble\string#1}}
\makeatother
\newcommand*{\meta}[1]{\textlangle\textsl{#1}\textrangle}
\newcommand*{\marg}[1]{\texttt{\{}\meta{#1}\texttt{\}}}
\newcommand*{\oarg}[1]{\texttt{[}\meta{#1}\texttt{]}}
\newcommand*{\parg}[1]{\texttt{(}\meta{#1}\texttt{)}}
\newcommand*{\BibTeX}{Bib\kern-0.08em\TeX\@\xspace}

\pdfstringdefDisableCommands{%
  \def\BibTeX{BibTeX\xspace}%
}

\addtokomafont{title}{\rmfamily}

\title{The \pkg{mslapa} package\thanks{This manual documents the \BibTeX style \pkg{mslapa.bst} and the accompanying package \pkg{mslapa.sty}.}}
\author{Michael S. Landy\thanks{\mail{landy@nyu.edu}}\and Clea F. Rees\and Nelson H. F. Beebe\thanks{\mail{beebe@plot79.math.utah.edu}}}
\date{5/1/95}


\begin{document}

\maketitle

\noindent
This software is based on previous APA formats as explained below. The
changes made to produce the \pkg{mslapa} format were originally made by Michael S.
Landy in 1994 and are public domain. The copyright and licensing of the file
as a whole also depends on the copyright and licensing of the original code.
Further details may be found below.\\
This statement added 2008/11/25 by Clea F. Rees in an attempt to make the
situation a little clearer following communication with Michael S. Landy.

This pair of files automates most (although not all) of the bibliographic
and citation formatting of APA (American Psychological Association) style.
The various citations formats that you want are things like:
%
\begin{enumerate}
\item \label{item:1} `in the work of Landy (1985, 1989a,b)' 
\item \label{item:2} `in the work of Landy, Dosher and Sperling (1989)' 
\item \label{item:3} `in the work of Landy and colleagues (1989)' 
\item \label{item:4} `it has been shown (Landy, 1989) that \ldots' 
\item \label{item:5} `it has been shown (Landy, 1989a, 1990; Sperling, 1991) that \ldots' 
\item \label{item:6} `it has been shown (Landy, Dosher \& Sperling, 1989a) that \ldots' 
\item \label{item:7} `it has been shown (Landy et al.,\ 1989a) that \ldots' 
\item \label{item:8} `it has been shown (e.g.,\ Landy et al.,\ 1989a) that \ldots' 
\item \label{item:9} `it has been shown (see, e.g.,\ Landy et al.,\ 1989a, for discussion) \ldots' 
\end{enumerate}

\pagebreak[2]

These are achieved by a number of citation macros:
%
\begin{itemize}
\item \cmd{\cite}\parg{PRENOTE}\oarg{NOTE}\verb+{+\meta{KEY1}\verb+,+\meta{KEY2}\verb+}+ -- for cases \ref{item:4}, \ref{item:5} and~\ref{item:6}
\item \cmd{\shortcite}\parg{PRENOTE}\oarg{NOTE}\verb+{+\meta{KEY1}\verb+,+\meta{KEY2}\verb+}+ -- for case \ref{item:7}, \ref{item:8} and~\ref{item:9}
\item \cmd{\citeyear}\parg{PRENOTE}\oarg{NOTE}\verb+{+\meta{KEY1}\verb+,+\meta{KEY2}\verb+}+ -- for cases \ref{item:1}, \ref{item:2} and~\ref{item:3}
\end{itemize}
%
Note that \meta{PRENOTE} will be put in the parentheses preceding the citations
(and followed by a blank), and \meta{NOTE} will follow the citations (preceded by
a comma).  Both \meta{PRENOTE} and \meta{NOTE} are (independently) optional arguments.
Thus, number~\ref{item:8} above was produced by
\verb+\shortcite(e.g.,){LANDYETAL89A}+,
and number~\ref{item:9} was produced by
\verb+\shortcite(see, e.g.,)[for discussion]{LANDYETAL89A}+.

\cmd{\shortcite} will use `et al.'\ for all citations of 3 or more authors.  If
you have a string of citations with more than one citation of 3 or more
authors, and you want some of those to have et al. and others to be spelled
out, then you will have to use \cmd{\nocite}\verb+{+\meta{KEY1}\verb+,+\meta{KEY2}\verb+}+ and format the citation
yourself by hand:
%
\begin{flushleft}
  `\texttt{%
    it has been shown (Landy, Dosher \cmd{\&} Sperling, 1989a; Sperling et al.,
    1990\cmd{\nocite}\{LANDETAL89A,SPERETAL90\}) that \ldots%
  }'
\end{flushleft}

There are also macros which insert only the author text without the year
(\cmd{\fullciteA} and \cmd{\shortciteA}), although it's not clear to me where one would
want to use that.  If you want to format things in any other way, you are
stuck formatting the citation yourself, e.\,g.:
%
\begin{verbatim}
In 1989, Landy and colleagues\nocite{LANDETAL89} discovered...
\end{verbatim}

The style macros for \texttt{cite}/\texttt{shortcite}/\texttt{citeyear} will lay things out as
described above, so that if two citations in a row have the same authors,
the authors are listed once, and if the same year as well, only the letter
names are added (as in \texttt{1989a,b})\@.  The algorithm recognizes that the form
`Landy et al.'\ can refer to any articles with 3 or more authors beginning
with \textsl{Landy}, and hence all such cited articles require letters after the year
(just in case you happen to cite them using `et al.')\@.  The letters are
added in such a way that the citation letters for a given \texttt{shortauthor}/\texttt{year}
combination appear in order in the references section.  Also, in the reference
section, and author such as `\verb+von Helmholtz+' will be alphabetized under \textsl{H}
and printed as `von Helmholtz, H. (1866).\ \ldots'\@.  If you prefer to
alphabetize this under \textsl{V} you can either put the name in braces \verb+{von Helmholtz}+
in your bibliography file (to make it look like a last name), or you can
alter the function `\texttt{sort.format.names}' below as indicated in the comment to
be found there.

\pagebreak[2]

Things this style won't do for you:
%
\begin{itemize}
\item Citations come out in the order you put them in the argument
  to \texttt{cite}\slash\texttt{shortcite}\slash\texttt{citeyear}\@.  So, if you want them sorted a
  particular way (alphabetically within year or vice versa), you have
  to do that yourself.
\item APA specifies that full author order should be used the first time
  a reference is cited, and \textit{et al.}\ (for 3 or more authors) thereafter.
  You have to do this yourself by using \cmd{\cite} or \cmd{\citeyear} the first
  time and \cmd{\shortcite} thereafter.
\end{itemize}

\end{document}
