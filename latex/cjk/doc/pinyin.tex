\documentclass[pagesize=auto, fontsize=12pt, DIV=10, parskip=half]{scrartcl}

\usepackage{fixltx2e}
\usepackage{etex}
\usepackage{lmodern}
\usepackage[T1]{fontenc}
\usepackage{textcomp}
\usepackage{hologo}
\usepackage{microtype}
\usepackage{hyperref}

\newcommand*{\mail}[1]{\href{mailto:#1}{\texttt{#1}}}
\newcommand*{\pkg}[1]{\textsf{#1}}
\newcommand*{\cs}[1]{\texttt{\textbackslash#1}}
\makeatletter
\newcommand*{\cmd}[1]{\cs{\expandafter\@gobble\string#1}}
\makeatother
\newcommand*{\opt}[1]{\texttt{#1}}

\addtokomafont{title}{\rmfamily}

\title{The \pkg{pinyin} package\thanks{This manual corresponds to \pkg{pinyin.sty}~v4.8.2, dated~29-Dec-2008.}}
\author{Werner Lemberg\thanks{\mail{wl@gnu.org}}}
\date{29-Dec-2008}


\begin{document}

\maketitle

This style file (which can be also used under \hologo{plainTeX}) enables the input
of pinyin syllables with tones.

Say
%
\begin{verbatim}
    \usepackage{pinyin}
\end{verbatim}
%
to load all pinyin macros under \hologo{LaTeX2e}; say
%
\begin{verbatim}
\input pinyin.sty
\end{verbatim}
%
under \hologo{plainTeX}.

An example explains best how to input pinyin:
%
\begin{verbatim}
    \Wo3 \hen3 \xi3\huan1 \chi1 \Zhong1\guo2 \cai4.
\end{verbatim}
%
Note there is no fifth tone marker in pinyin (Zhuyinfuhao uses a dot to
indicate the fifth tone; on the other hand no marker is used for the first
tone). Nevertheless you can say e.\,g.\@, \verb+\ne5+ to get the syllable `ne' without a
tone.

There are some special cases: 

\begin{itemize}
\item
  use `v' instead of `u umlaut' in pinyin syllables (these are \cmd{\lv},
  \cmd{\lve}, \cmd{\nv}, \cmd{\nve} and its uppercase forms). Example:
  % 
\begin{verbatim}
    \nv3'\er2       daughter
\end{verbatim}
  % 
  The appearance of u umlaut with additional tone markers has been
  tested with the standard \textsc{cm}, \textsc{ec}, and PostScript fonts.

\item use \cmd{\Long} and \cmd{\LONG} instead of \cmd{\long} and \cmd{\Long} (which you would
  expect): \cmd{\long} is a very important internal \hologo{TeX} command. Many
  packages would fail if we redefined \cmd{\long}.
\end{itemize}


\minisec{Problems:}

The following macros are redefined if you load pinyin.sty:
%
\begin{verbatim}
  \a, \chi, \cong, \ding, \ge, \hang, \le, \min, \mu, \ne, \ni, \nu,
  \o, \O, \pi, \Pi, \Re, \tan, \xi, \Xi.
\end{verbatim}
%
They are available as \cmd{\PY...} (\cmd{\PYchi}, \cmd{\PYcong}, etc.).

If this is not enough, you can say \cmd{\PYdeactivate} to restore the original
definitions (and reactivate these syllables with \cmd{\PYactivate}).

In case you use the \pkg{hyperref} package earlier than version~6.75a
(2006-Feb-12) together with the `\opt{hpdftex}' driver you should load
\pkg{pinyin.sty} \emph{after} \pkg{hyperref.sty} (contrary to what the \pkg{hyperref} manual
says). Reason is that \cmd{\ding} is defined in \pkg{pifont.sty} which is
automatically loaded by \pkg{hpdftex.def} -- the latest \pkg{hyperref} version no
longer uses \pkg{pifont.sty}.

\end{document}
