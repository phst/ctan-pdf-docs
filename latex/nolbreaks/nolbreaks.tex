\documentclass[pagesize=auto, fontsize=14pt, DIV=10, parskip=half]{scrartcl}

\usepackage{fixltx2e}
\usepackage{etex}
\usepackage{xspace}
\usepackage{lmodern}
\usepackage[T1]{fontenc}
\usepackage{textcomp}
\usepackage{microtype}
\usepackage{hyperref}

\newcommand*{\pkg}[1]{\textsf{#1}}
\newcommand*{\cs}[1]{\texttt{\textbackslash#1}}
\makeatletter
\newcommand*{\cmd}[1]{\cs{\expandafter\@gobble\string#1}}
\makeatother
\newcommand*{\opt}[1]{\texttt{#1}}
\newcommand*{\meta}[1]{\textlangle\textsl{#1}\textrangle}
\newcommand*{\marg}[1]{\texttt{\{}\meta{#1}\texttt{\}}}

\addtokomafont{title}{\rmfamily}

\title{The \pkg{nolbreaks} package\thanks{This manual corresponds to \pkg{nolbreaks}~v1.0, dated~2002/09/19.}}
\author{Donald Arseneau}
\date{2002/09/19}


\begin{document}

\maketitle

Use \cmd{\nolbreaks}\marg{some text} to prevent linebreaks in \meta{some text}.
This has the advantage over \verb+\mbox{}+ that glue (rubber space) 
remains flexible.  It has the disadvantage of not working in 
all cases!  Most common cases are handled here (\cmd{\linebreak} is 
disabled, for example) but spaces hidden in macros or \verb+{ }+
can still create break-points.

Large pieces of text with no breaks can cause problems with
paragraph justification.  Giving the package option \opt{[ragged]}
allows a line before the unbreakable text to be cut short.

You should declare \cmd{\sloppy} in your document.

\end{document}
