\documentclass[pagesize=auto]{scrartcl}

\usepackage{fixltx2e}
\usepackage{etex}
\usepackage{lmodern}
\usepackage[T1]{fontenc}
\usepackage{textcomp}
\usepackage[utf8]{inputenc}
\usepackage{microtype}
\usepackage{hyperref}

\newcommand*{\mail}[1]{\href{mailto:#1}{\texttt{#1}}}
\newcommand*{\pkg}[1]{\textsf{#1}}
\newcommand*{\cs}[1]{\texttt{\textbackslash#1}}
\makeatletter
\newcommand*{\cmd}[1]{\cs{\expandafter\@gobble\string#1}}
\makeatother
\newcommand*{\meta}[1]{\textlangle\textsl{#1}\textrangle}
\newcommand*{\marg}[1]{\texttt{\{}\meta{#1}\texttt{\}}}

\addtokomafont{title}{\rmfamily}

\title{The \pkg{theoremref} package}
\author{Emil Jeřábek\thanks{\mail{jerabek@math.spam.cz} (replace \texttt{spam} with \texttt{cas})}}
\date{2008/05/13}


\begin{document}

\maketitle


\section{Introduction}

So, you are writting up your math paper or thesis. There is the
all-important Lemma~3.14 referenced from a dozen places throughout
the paper, but you have just realized that its proof requires another
lemma which should go before Lemma~3.14, hence all the references to
Lemma~3.14 have to shift to Lemma~3.15. Now that's quite simple to do:
you just insert the new lemma, and \LaTeX\ relabels the references for you
behind the scenes. Good.

Then you decide that Lemma~3.15 is, in fact, so important that you
better call it a Theorem rather than just a Lemma, hence all the
references to Lemma~3.15 should change to Theorem 3.15. Now that's
quite simple to do: you just replace the name of the theorem
environment of~3.15, and \LaTeX\ relabels the references for you behind
the scenes\dots\ uhhh, except that it does not, actually. You have to go
through the paper and manually change every occurrence of
\verb|Lemma~\ref{main}| to \verb|Theorem~\ref{main}|. Your editor software may
help with automatic text replacement, but that's no good by itself,
you unfortunately still have to watch out for cases like
\verb|Lemmas \ref{baz-quux} and~\ref{main}|. Needless to say, the whole business
is rather error-prone, and very annoying, especially if you later
decide that 3.15 is, after all, a Lemma, with the prospect of changing
all those references back again.

The \pkg{theoremref} package is designed to fill this gap in the \LaTeX\ %
automatic reference system. It provides variants of the \cmd{\label} and
\cmd{\ref} commands which automatically supply the correct theorem
environment name into a reference, thus avoiding all the hassle
described above.


\section{Basic usage}

\renewcommand*{\labelenumi}{(\theenumi)}
\begin{enumerate}
\item Put \verb|\usepackage{theoremref}| anywhere in the document preamble.
  \pagebreak[2]

\item For each theorem which you intend to use in the new system, declare
  its label by \cmd{\thlabel}\marg{foobar}. For example,
%
\begin{verbatim}
  \begin{Lem}\thlabel{exact}
    Every projective formula is exact.
  \end{Lem}
\end{verbatim}

\item There are three commands available for producing references to
    your theorems: \cmd{\ref}\marg{foo} gives the number, as usual;
    \cmd{\thref}\marg{foo} gives the theorem name followed by number; and
    \cmd{\thnameref}\marg{foo} gives just the theorem name. Example:

    \begin{tabular}{l@{ $\to$ }l@{}}
      \verb|\ref{exact}|       & \verb|5.7.7|       \\
      \verb|\thref{exact}|     & \verb|Lemma~5.7.7| \\
      \verb|\thnameref{exact}| & \verb|Lemma|
    \end{tabular}

    The \cmd{\thref} command can handle an arbitrary number of arguments,
    separated by commas:

    \begin{tabular}{l@{ $\to$ }l@{}}
      \verb|\thref{exact,zorn,main}| & \verb|Lemmas 5.7.7 and~2.1.5 and Theorem~6.6.4|
    \end{tabular}
\end{enumerate}


\section{Options}

\begin{itemize}
\item \verb|\usepackage[lowercase]{theoremref}| will set your theorem
  references in lowercase, e.\,g., ``lemma 5.7.7''. Note that there is
  no provision for capitalization at the beginning of a sentence.

\item \sloppy \verb|\usepackage[reftex]{theoremref}| provides an alternative
  interface to the main commands: you can say \cmd{\th}\cmd{\label}\marg{foo} and
  \cmd{\th}\cmd{\ref}\marg{foo} instead of \cmd{\thlabel}\marg{foo} and \cmd{\thref}\marg{foo}.
  The effect is that the labelling and referencing commands are
  correctly recognized as such by the Emacs reftex package (and,
  presumably, other similar editing tools).
\end{itemize}

\fussy


\section{Caveats}

\begin{itemize}
\item The \cmd{\thlabel} command reuses the slot for page number in the \texttt{.aux}
  file. This should do no harm, as the latter is generally useless, I've
  never seen anybody refer to a numbered theorem by its page number. But
  to be on the safe side, here's an explicit warning: if you declare \meta{foo}
  by \cmd{\thlabel}\marg{foo}, you cannot use \cmd{\pageref}\marg{foo} to get its page
  number (it will actually give the theorem name). If you absolutely
  need both, you can declare two labels for the same theorem:
  \cmd{\label}\marg{foo-page}\cmd{\thlabel}\marg{foo-name}.

\item If \cmd{\thref}\marg{bar} gives you a cryptic result like ``12 3.15''
  instead of the theorem name, you probably forgot to use \cmd{\thlabel}
  instead of \cmd{\label} (see the previous point for explanation). Note
  that after you switch from one to the other, you may need to \TeX\ the
  file twice in order for the change to propagate to the \texttt{.aux} file and
  back.

\item The new referencing commands only work for theorem environments
  declared using the \cmd{\newtheorem} command. You \emph{cannot} use them for
  e.\,g.\ tables, figures, equations, sections, and other environments.

\item The implementation depends on some internals of the theorem
  typesetting macros. It is compatible with the \pkg{theorem} and
  \pkg{amsthm} packages, as well as the default theorem system in base
  \LaTeX. It may fail for other theorem-like packages.
\end{itemize}


\section{License}

The \pkg{theoremref} package is dually licensed under GPL or LPPL at your
option:

Copyright © 2008 Emil Jeřábek

This package may be distributed and/or modified under the
conditions of the \LaTeX\ Project Public License, either version 1.3
of this license or (at your option) any later version.
The latest version of this license is in
\url{http://www.latex-project.org/lppl.txt}
and version~1.3 or later is part of all distributions of \LaTeX\ %
version~2003/12/01 or later.

This package is free software; you can redistribute it and/or modify
it under the terms of the GNU General Public License as published by
the Free Software Foundation; either version~2 of the License, or
(at your option) any later version.

This package is distributed in the hope that it will be useful,
but WITHOUT ANY WARRANTY; without even the implied warranty of
MERCHANTABILITY or FITNESS FOR A PARTICULAR PURPOSE.  See the
GNU General Public License for more details.

You should have received a copy of the GNU General Public License
along with this program; if not, write to the Free Software
Foundation, Inc., 51 Franklin Street, Fifth Floor, Boston, MA  02110-1301  USA

\end{document}
