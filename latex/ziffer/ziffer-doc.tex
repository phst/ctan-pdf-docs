\documentclass[pagesize=auto]{scrartcl}

\usepackage{fixltx2e}
\usepackage{etex}
\usepackage{lmodern}
\usepackage[T1]{fontenc}
\usepackage{textcomp}
\usepackage[utf8]{inputenc}
\usepackage{microtype}
\usepackage{hyperref}

\newcommand*{\mail}[1]{\href{mailto:#1}{\texttt{#1}}}
\newcommand*{\pkg}[1]{\textsf{#1}}
\newcommand*{\cs}[1]{\texttt{\textbackslash#1}}
\makeatletter
\newcommand*{\cmd}[1]{\cs{\expandafter\@gobble\string#1}}
\makeatother
\newcommand*{\ZifferStrich}{\textrm{-\kern0.1em-}}

\addtokomafont{title}{\rmfamily}

\title{The \pkg{ziffer} package\thanks{This manual corresponds to \pkg{ziffer.sty}~v2.1, dated~2002.}}
\author{M. Väth\thanks{\mail{vaeth@mathematik.uni-wuerzburg.de}}}
\date{2002}


\begin{document}

\maketitle

\noindent
This package provides the macros
%
\begin{flushleft}
  \begin{tabular}{@{}ll@{}}
    \cmd{\ZifferAn}  & (equivalent with \verb+\ZifferPunktAn \ZifferStrichAn +) \\
    \cmd{\ZifferAus} & (equivalent with \verb+\ZifferPunktAus\ZifferStrichAus+) \\
    \cmd{\ZifferPunktAn}                                                        \\
    \cmd{\ZifferPunktAus}                                                       \\
    \cmd{\ZifferStrichAn}                                                       \\
    \cmd{\ZifferStrichAus}
  \end{tabular}
\end{flushleft}
%
If \cmd{\ZifferAn} was used, then in numbers in math-mode the following happens:
%
\begin{flushleft}
  \begin{tabular}{@{}rl@{}}
    `\texttt{.}'  & is used as a `one-thousand separator' as common in Germany \\
    `\texttt{,}'  & is used as a decimal separator as common in Germany        \\
    `\texttt{-{}-}' & generates a \ZifferStrich\ with better spacing (e.\,g.\ in 1.000,\ZifferStrich)
  \end{tabular}
\end{flushleft}
%
You may still use the symbols `\texttt{.}' `\texttt{,}' and `\texttt{-}' in other context in math mode.

The first two conversions are switched on by default, the last conversion has
to be switched on explicitly (this has been changed in v2.1):
You may switch on/off the conversions with
%
\begin{flushleft}
  \begin{tabular}{@{}ll@{}}
    \cmd{\ZifferPunktAn}/\cmd{\ZifferPunktAus}   & (for the first two conversions) respectively \\
    \cmd{\ZifferStrichAn}/\cmd{\ZifferStrichAus} & (for the last conversion).
  \end{tabular}
\end{flushleft}

The reason why the last conversion is not on by default (and why it might be
necessary to [temporarily] switch off a conversion) is that certain packages
(in particular, newer versions of \pkg{amsmath}) cause problems with it.

There were some discussions which should be the correct output of the
above conversions (in particular, of `\texttt{-{}-}').
If you do not like my choice: You may customize (i.\,e.\ redefine) the macros
\cmd{\ZifferLeer} and \cmd{\ZifferStrich}
after loading the package to produce the `one-thousand separator'
respectively the `\ZifferStrich'.

All above command-names are in German, because I had expected that this
package is only needed for German texts. Meanwhile, I learned that also
other countries use this strange convention for numbers. However, for
downward compatibility, I decided to keep the name conventions anyway
(after all, the name of the package is in German anyway).

\end{document}
