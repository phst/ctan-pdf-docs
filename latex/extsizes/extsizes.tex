\documentclass[DIV=9, headings=normal, pagesize=auto]{scrartcl}

\usepackage{fixltx2e}
\usepackage{etex}
\usepackage{lmodern}
\usepackage[T1]{fontenc}
\usepackage{textcomp}
\usepackage{microtype}
\usepackage[unicode=true]{hyperref}

\newcommand*{\mail}[1]{\href{mailto:#1}{\texttt{#1}}}
\newcommand*{\pkg}[1]{\textsf{#1}}
\newcommand*{\cls}[1]{\textsf{#1}}
\newcommand*{\cs}[1]{\texttt{\textbackslash#1}}
\makeatletter
\newcommand*{\cmd}[1]{\cs{\expandafter\@gobble\string#1}}
\makeatother
\newcommand*{\opt}[1]{\texttt{#1}}

\addtokomafont{title}{\rmfamily}
\addtokomafont{section}{\rmfamily}

\title{The \cls{extsizes} classes}
\author{James Kilfiger, \mail{james.kilfiger@gmail.com}\and Wolfgang May, \mail{may@informatik.uni-goettingen.de}}
\date{}


\begin{document}

\maketitle

\noindent
The standard \LaTeX\ classes (\cls{article}, \cls{report} etc) support ten, eleven and
twelve point text. These are the commonest sizes used in publishing.
However, for certain applications there may be a need for other sizes.
The \cls{extsizes} classes (\cls{extarticle}, \cls{extreport}, \cls{extbook}, \cls{extletter}, and
\cls{extproc}) provide support for sizes eight, nine, ten, eleven, twelve,
fourteen, seventeen and twenty points.

The \cls{extsizes} classes and class options were first written by Wolfgang
May, by adapting the standard \LaTeX\ classes. James Kilfiger
made some modifications and rewrote the size options. 


\section{Should you be using \cls{extsizes}?}

Don't use \cls{extsizes} just because you think its cool, or because you think
the font looks too small on the screen.  You should have a clear reason
why 10, 11 or 12\,pt text is not suitable for you.  Also the \pkg{extsizes}
package is not suitable for creating oversize pages for scaling by a
printer or photocopier, this can be done with the the \pkg{geometry} package
and the \opt{mag} option (another of my hacks I'm afraid).  Good reasons for
using the \pkg{extsizes} package might include conforming to requirements set
by an examining institution, or making a large print copy for use by the
partially sighted.


\section{How to install \cls{extsizes}.}

You should place all the files in `a place where \TeX\ can find them'.
Examples of where \TeX\ looks for files include the \texttt{.../texmf/tex/latex}
directory tree, a local \texttt{texmf} tree, anywhere specified in a \texttt{TEXINPUTS}
environment variable or the same directory as your \LaTeX\ documents.  
You should then refresh the file name database. This is done with a
command `\texttt{texhash}' on te\TeX\ distributions and from the start menu with
Mik\TeX\@.   For other distributions of \TeX\ read the manual to see if this
step is required.


\section{How to use \cls{extsizes}.}

Your documentclass command should look like:
%
\begin{verbatim}
\documentclass[14pt]{extreport}
\end{verbatim}
%
or
%
\begin{verbatim}
\documentclass[9pt]{extarticle}
\end{verbatim}
%
The sizes available are \opt{8pt}, \opt{9pt}, \opt{10pt}, \opt{11pt}, \opt{12pt}, \opt{14pt}, \opt{17pt}, and \opt{20pt}.
There should be no need to change any other part of your document.

There is also a package, \pkg{extsizes.sty}, which can be used with
nonstandard document classes.  But it cannot be guaranteed to work with
any give class.  Don't use it at the same time as one of the \cls{extsizes}
classes. It takes as package options \opt{8pt}\,--\,\opt{20pt}.  This comes from an idea
of Hans Steffani.


\section{Compatibility mode broken.}

You cannot use these classes in `compatibility mode', nor can they be
used with \LaTeX2.09.  Therefore \verb+\documentstyle[20pt]{extarticle}+ won't
work.  Compatibility mode should not be used for new documents.


\section{Bad line breaking.}

If you are using 20\,pt text \TeX\ will probably have difficulty finding
good line breaks, so you will get warnings about overfull hboxes.  
If this is distracting you may put \cmd{\sloppy} in the preamble of your
document, but it is better to try to help \TeX\ to find good line breaks
by inserting \cmd{\-} or rewriting short sections.


\section{Warnings about Font substitutions.}

The large class options use some very large fonts. Up to about 50\,pt for
the \opt{20pt} class options.  You may find this causes problems with with
if these fonts are not declared to be available by a \cmd{\DeclareFontShape}
command.  You should either use fonts like \pkg{Times} which are usually
available in all sizes, or redeclare the fonts.  There is an example of
this in the \cls{extsizes} classes for \texttt{cmr}.  You should also read \pkg{fntguide}.

\bigskip

\begin{flushleft}
James\\
\quad\mail{james.kilfiger@gmail.com}\\
Wolfgang\\
\quad\mail{may@informatik.uni-goettingen.de}\\
(in case the addresses become invalid in the future, 
look us up in the Web)
\end{flushleft}

\end{document}












