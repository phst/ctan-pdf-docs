\documentclass[pagesize=auto]{scrartcl}

\usepackage{fixltx2e}
\usepackage{etex}
\usepackage{lmodern}
\usepackage[T1]{fontenc}
\usepackage{textcomp}
\usepackage{amstext}
\usepackage{booktabs}
\usepackage{microtype}
\usepackage{hyperref}

\newcommand*{\mail}[1]{\href{mailto:#1}{\texttt{#1}}}
\newcommand*{\pkg}[1]{\textsf{#1}}
\newcommand*{\cs}[1]{\texttt{\textbackslash#1}}
\makeatletter
\newcommand*{\cmd}[1]{\cs{\expandafter\@gobble\string#1}}
\makeatother
\newcommand*{\meta}[1]{\textlangle\textsl{#1}\textrangle}
\newcommand*{\marg}[1]{\texttt{\{}\meta{#1}\texttt{\}}}

\addtokomafont{title}{\rmfamily}

\title{The \pkg{relsize} package\thanks{This manual corresponds to \pkg{relsize.sty}~v3.1, dated~July 4, 2003.}}
\author{%
  by Donald Arseneau\thanks{\mail{asnd@triumf.ca},  Vancouver, Canada}%
  \and origianally based on \pkg{smaller.sty} by Bernie Cosell\thanks{\mail{cosell@WILMA.BBN.COM}},%
  \and and combined with the version by Matt Swift\thanks{\mail{swift@bu.edu}}.%
}   
\date{July 4, 2003}


\begin{document}

\maketitle

\noindent
It is frequently the case that something should be typeset somewhat larger
or smaller than the surrounding text, whatever that size happens to be.
Specifying such sizing commands explicitly makes it difficult to modify the
font sizes of a document at a later time, and makes it hard to write macros
that work at arbitrary sizes.  \pkg{Relsize.sty} defines several commands for \LaTeX\ %
to set font sizes relative to the current size.

To refresh your memory, the font sizing commands in \LaTeX\ are, in order:
\cmd{\tiny}, \cmd{\scriptsize}, \cmd{\footnotesize}, \cmd{\small}, \cmd{\normalsize}, \cmd{\large}, \cmd{\Large},
\cmd{\LARGE}, \cmd{\huge}, \cmd{\Huge} (package \pkg{moresize} adds \cmd{\ssmall} and \cmd{\HUGE}).  The main 
new command provided by \pkg{relsize.sty} is \cmd{\relsize}, which takes one (positive
or negative) number as its argument; the number specifies how many ``steps''
by which to change the font size, where each step is a scaling factor of~%
1.2, corresponding to the usual difference between the size commands.  For
example, if \verb|{\relsize{-2} smaller}| appears in normal sized text, the word
``smaller'' is printed in script size type.  If the same command appears in a 
\cmd{\Large} section title, then ``smaller'' is printed in normal size. 

\begin{sloppypar}
  There are also the commands \cmd{\larger} and \cmd{\smaller}, which normally change the
  font size by one step in the obvious direction; \cmd{\larger} is an abbreviation
  for \verb|\relsize{+1}|, and \cmd{\smaller} is an abbreviation for \verb|\relsize{-1}|. For
  example, \verb+{\large... \larger{WOW!}}+ prints ``WOW'' in \cmd{\Large} type.  You can
  also specify bigger steps as an optional argument for \cmd{\larger} and \cmd{\smaller}: 
  \verb+\larger[3]+ is equivalent to \verb+\relsize{3}+; \verb+\smaller[2]+ is \verb+\relsize{-2}+.
  Both \cmd{\larger} and \cmd{\smaller} accept negative arguments, but please don't make
  things so obscure!  If you want to change size by several steps it is 
  much better to give an increment than to string several \cmd{\larger} commands
  together; i.\,e.\@, \verb+\relsize{3}+ or \verb+\larger[3]+, but not \verb+\larger\larger\larger+.
\end{sloppypar}

All of the \cmd{\relsize}, \cmd{\larger}, and \cmd{\smaller} commands are ``switches'' just
like the regular sizing commands.  That is, they change the size for all
following text until the scope is ended by a closing brace.  There are
alternate versions called \cmd{\textlarger} and \cmd{\textsmaller} that take some text
as an argument and apply the size change to only that text:
\verb+\textlarger{big}+.

Using the number of ``magnification steps'' to indicate font size can
be confusing to some people, and limiting in certain uses.  There are 
commands with syntax \cmd{\text\-scale}\marg{factor}\marg{text} and \cmd{\relscale}\marg{factor} to
select the size based on a scale factor, like \verb+{0.75}+.

If the size requested is too small or too large, a warning is given, and
the size will only change as far as appropriate, typically \cmd{\tiny} or \cmd{\Huge}.
These limits are controlled by the commands \cmd{\RSsmallest} and \cmd{\RSlargest}, which
get set automatically when \pkg{relsize.sty} is loaded, but you can redefine them
to other length values: \verb+\renewcommand\RSlargest{50pt}+.

Fine point:  The combination \cmd{\relsize}\marg{n}\cmd{\relsize}\texttt{\{}$-\text{\meta{n}}$\texttt{\}} is not guaranteed to 
restore the current font size!  That is because the increment \meta{n} may be
enough to overflow the range of sizes.  Say you are in \cmd{\huge} text already,
and you do \verb+\relsize{4}+.  There is nothing bigger than \cmd{\Huge} so that is the
size you get.  Then an ensuing \verb+\relsize{-4}+ will take four steps smaller
and change the size to \cmd{\large}.  You should use grouping to undo relative 
size changes because it is unsafe to counteract one change with an ``equal''
change in the opposite direction.

Typically, the font-size commands do not select fonts at regular size
intervals.  \cmd{\relsize} (and the others) will select, and execute, the command
for the size closest to the desired size.  Then, if the relative difference 
from the target size is more than \cmd{\RSpercentTolerance} a further font-size
selection is made.  By~default, \cmd{\RSpercentTolerance} is an empty macro, which
causes automatic selection: ``30'' (30\,\%) when the current ``fontshape'' definition
is composed of only discrete sizes, and ``5'' when the fontshape definition
covers ranges of sizes.  The higher setting for discrete fonts ensures
only the pre-defined sizes get used.  (By~default \LaTeX\ uses Computer
Modern fonts at discrete sizes; you get full coverage of sizes by using
\verb+\usepackage{type1cm}+\@.)  For special uses, or when the font shape definitions 
are not parsed properly, you can redefine the percent tolerance: 
\verb+\renewcommand\RSpercentTolerance{10}+.

\enlargethispage{-5\baselineskip}

All of the commands described above are text commands; they cannot be used
in math mode.  There are special \cmd{\mathsmaller} and \cmd{\mathlarger} commands
provided, but these do not use the same sizes that the text versions use.
Instead, they step between the usual math ``styles'' which you can explicitly
set using the commands \cmd{\displaystyle}, \cmd{\textstyle}, \cmd{\scriptstyle}, and
\cmd{\scriptscriptstyle} [see Lamport, \LaTeX/Manual (1st~ed, p.\,54); GMS The \LaTeX\ %
Companion, p.\,255]\@.  However, the \cmd{\mathlarger} command will also increase the
size beyond regular \cmd{\displaystyle} by selecting a larger regular font size
(using \cmd{\larger})\@.  (Yes, this is a kludge, and doesn't work very well, but 
it can still be useful.)  If you want to use this to create big integral 
signs, then you must also load the package \pkg{exscale} so that math symbols
can change size.  The sizes selected are:
%
\begin{flushleft}
  \small
  \begin{tabular}{@{}lll@{}}
    \toprule
    Current style            & \cmd{\mathsmaller} gives        & \cmd{\mathlarger} gives                     \\
    \cmidrule(r){1-1} \cmidrule(lr){2-2} \cmidrule(l){3-3}
    \cmd{\displaystyle}      & \cmd{\textstyle} (almost same)  & \cmd{\displaystyle} in a \cmd{\larger} font \\
    \cmd{\textstyle}         & \cmd{\scriptstyle}              & \cmd{\displaystyle} (almost same)           \\
    \cmd{\scriptstyle}       & \cmd{\scriptscriptstyle}        & \cmd{\textstyle}                            \\
    \cmd{\scriptscriptstyle} & \cmd{\scriptscriptstyle} (same) & \cmd{\scriptstyle}                          \\
    \bottomrule
  \end{tabular}
\end{flushleft}
%
For example, try \verb+$\frac{\mathlarger{E}}{E}$+.  Note that, for most symbols,
\cmd{\display\-style} and \cmd{\textstyle} are the same size, so \verb+$N \mathlarger{N}$+
gives two identical $N$'s, but \cmd{\sum} and \cmd{\int} do get bigger in display style: 
\verb+$\int\mathlarger{\int}$+, and fractions are treated differently too: 
\verb+$\frac{1}{2} \mathlarger{\frac{1}{2}}$+.  As you might have guessed, 
\cmd{\mathlarger} and \cmd{\mathsmaller} should only be used in math mode.

These commands will attempt to attach superscripts and subscripts to the 
symbol as if the symbol were used alone: \verb+$\mathlarger{\int}_{0}^{\infty}$+
but math accents and the math spacing behave as if the symbol is enclosed
in braces (which it is).  Comparing 
%
\begin{flushleft}
  \footnotesize
\begin{verbatim}
\[
| \times | {\times} | \mathlarger{\times} | \mathbin{\mathlarger{\times}} |
\] 
\end{verbatim}
\end{flushleft}
%
shows that operators must be explicitly declared to use the right operator
type (\cmd{\mathrel}, \cmd{\mathbin}, \cmd{\mathop}) to get the correct spacing.

Due to their oddities, the math larger/smaller commands should not be 
trusted blindly, and they will not be useful in every instance.

\end{document}
