\documentclass[DIV=9, pagesize=auto]{scrartcl}

\usepackage{xspace}
\usepackage{shortvrb}

\newcommand*{\BibTeX}{Bib\kern-0.08em\TeX\@\xspace}
\newcommand*{\BigTeX}{Big\kern-0.08em\TeX\@\xspace}
\newcommand*{\ILaTeX}{I\kern0.03em\LaTeX\@\xspace}

\addtokomafont{title}{\rmfamily}

\MakeShortVerb{|}

\title{The \textsf{biblist} package\thanks{This manual corresponds to \textsf{biblist}~v1.2, dated~1992/01/13}}
\author{Joachim Schrod\thanks{Computer Science Department, Technical University of Darmstadt, Germany} \\ Email: \texttt{schrod@iti.informatik.th-darmstadt.de}}
\date{1992/01/13}


\begin{document}

\maketitle

\noindent

There is an often asked question: `How can I print a listing with all
entries of a \BibTeX database?'. The common answer is: `Use
|\nocite{*}|.'

Well, but this will not work for large \BibTeX databases unless you
use \BigTeX. This is because for each entry a \TeX\ macro is declared
and \TeX's pool size capacity (where macro names are stored) will be
exceeded soon.

\medskip

This style option is appropriate to create a typeset listing of a
(possibly large) \BibTeX input file. With such large files --~%
especially, if the cite keys are long~-- the needed string space is
often exceeded. Often a \BigTeX is available to circumstance this
problem, but with this style option each \TeX\ will do it.

You have to prepare a \LaTeX\ document which uses the |article|
style and the |biblist| style option. You may add almost all other
style options, as you wish, e.\,g., |twoside|, |german| (or other
language style options), |a4|, etc. This style option must be used
with a ragged bottom; this has the effect, that it cannot be used with
|twocolumn| or |multicol|.

You must issue a |\bibliography| tag which names all \BibTeX
databases which you want to print. You may issue a
|\bibliographystyle| tag to specify how \BibTeX will process its
databases. (In fact, you usually must issue it since the default
bibliography style is not available on most installations; see
below.) You may issue |\nocite| commands if you want to print only
selected entries from the databases.

\medskip

A ``bug'' you may encounter is that |\cite| tags within
\BibTeX entries will not be processed. Instead the cite key itself
will be printed. Note that this is not a bug, this is a feature! You
have to use |\nocite| tags for \emph{all} entries that shall be
included in the listing. If you do not give any |\nocite| tag at all,
a listing with all entries is created.

Note that this style option assumes that you use either \LaTeX~2.09
released after December 1991, \ILaTeX, or the \textsf{babel} system.
Particulary, it requires the definition of |\refname| which is supplied
by these systems. (Of course, you may supply this definition by
yourself.)

\medskip

The entries in the resulting listing are formatted as follows:
%
\begin{quote}
  \texttt{cite key}~\leaders\hbox{.}\hfill~\textsf{(Library info)}\\
  \null\quad Author(s).\\
  Title.\\
  Publication info.\\
  Notes.\\
  \null\quad{\footnotesize Annotation}
\end{quote}
%
I.\,e., an open format is used. Although this needs more space I think
the enhanced legibility pays back.

Note that you will not get the `Library info' and the `Annotation' in
the above format if your bibliography style does not supply this
information with the assumed markup. The parenthesis around the library
info are produced by this style option, not by \BibTeX.

Enjoy. btw, this style is supported since I use it myself :-)


\minisec{Changes:}
%
\begin{itemize}
\item Revision 1.2, 1992/01/13:
  Updated documentation, adapted to \LaTeX\ version of Dec 91.
  New \texttt{biblist.bst-dist} (\texttt{itibst.doc} version 1.3).
\item Revision 1.1, 1991/10/14:
  Initial revision
\end{itemize}

\end{document}
