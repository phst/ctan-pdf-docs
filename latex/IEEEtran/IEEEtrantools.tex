\documentclass[pagesize=auto]{scrartcl}

\usepackage{lmodern}
\usepackage[T1]{fontenc}
\usepackage{textcomp}
\usepackage{xcolor}
\usepackage{listings}
\usepackage{microtype}
\usepackage{hyperref}

\addtokomafont{disposition}{\rmfamily}

\lstset{language=[LaTeX]TeX, columns=fullflexible, basicstyle=\small\ttfamily, commentstyle=\color{gray}}

\title{The \textsf{IEEEtrantools} package}
\author{Michael Shell}
\date{2007/01/11}


\begin{document}

\maketitle

\tableofcontents

\bigskip

\noindent
This is a PDF version of \texttt{IEEEtrantools\_doc.txt} 2007/01/11 version 1.2
(Version 1.2 is based on the commands of version 1.7 of IEEEtran.cls) \\
This document is the user guide for the \textsf{IEEEtrantools.sty} package.

The \textsf{IEEEtrantools.sty} package provides several popular and unique
commands from the \textsf{IEEEtran.cls} class (version 1.7) file.

\textsf{IEEEtrantools.sty} should not be used with \textsf{IEEEtran.cls}.

For helpful tips, answers to frequently asked questions and other support,
visit the IEEEtrantools support page at my website: \\
\url{http://www.michaelshell.org/tex/ieeetran/tools/}

The latest version and documentation of IEEEtrantools can be obtained at: \\
\url{http://www.ctan.org/tex-archive/macros/latex/contrib/IEEEtran/}

Copyright \textcopyright\ 2002--2007 by Michael Shell \\
See
\url{http://www.michaelshell.org/}
for current contact information.


\section{Legal Notice:}

This code is offered as-is without any warranty either expressed or
implied; without even the implied warranty of MERCHANTABILITY or
FITNESS FOR A PARTICULAR PURPOSE!\@
User assumes all risk.
In no event shall IEEE or any contributor to this code be liable for
any damages or losses, including, but not limited to, incidental,
consequential, or any other damages, resulting from the use or misuse
of any information contained here.

All comments are the opinions of their respective authors and are not
necessarily endorsed by the IEEE.

This work is distributed under the \LaTeX\ Project Public License (LPPL)
(\url{http://www.latex-project.org/}) version 1.3, and may be freely used,
distributed and modified. A copy of the LPPL, version 1.3, is included
in the base \LaTeX\ documentation of all distributions of \LaTeX\ released
2003/12/01 or later.
Retain all contribution notices and credits.
\emph{Modified files should be clearly indicated as such, including
renaming them and changing author support contact information.}

File list of work: \texttt{IEEEtrantools.sty}, \texttt{IEEEtrantools\_doc.txt}


\section{Available package options}

e.\,g., \verb+\usepackage[retainorgcmds]{IEEEtrantools}+

\begin{description}
\item[retainorgcmds]
  Prevents \textsf{IEEEtrantools} from overriding existing \LaTeX\ commands.
  Currently, the only effect is to preserve the original definitions
  of \texttt{itemize}, \texttt{enumerate} and \texttt{description}. The \textsf{IEEEtran} versions are
  always available as \texttt{IEEEitemize}, \texttt{IEEEenumerate} and \texttt{IEEEdescription}
  and the original \LaTeX\ versions are always available as \texttt{LaTeXitemize},
  \texttt{LaTeXenumerate} and \texttt{LaTeXdescription}.
\end{description}


\section{1/2007 V1.2 (V1.7 of \texttt{IEEEtran.cls}) changes:}

\begin{enumerate}
\item Several commands and enviroments have depreciated in favor of
  replacements with IEEE prefixes to better avoid potential future name
  clashes with other packages. Legacy code retained to allow use of the
  obsolete forms, but with an warning message to the console during
  compilation:
  \verb+\IEEEPARstart+\\
  For IED lists:
  \verb+\IEEEiedlabeljustifyc+, \verb+\IEEEiedlabeljustifyl+,\\ \verb+\IEEEiedlabeljustifyr+,
  \verb+\IEEEnocalcleftmargin+, \verb+\IEEElabelindent+,\\ \verb+\IEEEsetlabelwidth+,
  \verb+\IEEEusemathlabelsep+
\item These commands/lengths now require the IEEE prefix and do not have
  legacy support: \verb+\IEEEnormaljot+.\\
  For IED lists: \verb+\ifIEEEnocalcleftmargin+, \verb+\ifIEEEnolabelindentfactor+,\\
  \verb+\IEEEiedlistdecl+, \verb+\IEEElabelindentfactor+
\item \verb+\normalsizebaselineskip+ no longer provided.
\item New \verb+\IEEEPARstart+ controls:
  \verb+\IEEEPARstartHEIGHTTEXT+,\\ \verb+\IEEEPARstartFONTSTYLE+, \verb+\IEEEPARstartCAPSTYLE+,\\
  \verb+\IEEEPARstartWORDFONTSTYLE+, \verb+\IEEEPARstartWORDCAPSTYLE+,\\
  \verb+\IEEEPARstartHOFFSET+, \verb+\IEEEPARstartITLCORRECT+\\
  and the (output) length \verb+\IEEEPARstartletwidth+.
\item Provide for an optional argument to \verb+\bstctlcite+ to provide a way to
  specify a different aux file. Define \verb+\bstctlcite+ even if it has already
  been defined.
\end{enumerate}


\section{11/2002 V1.1 (v1.6b of \texttt{IEEEtran.cls}) changes:}

\begin{enumerate}
\item In addition to the IEEE IED lists, the original LaTeX IED style list
  environments are now preserved as \texttt{LaTeXitemize}, \texttt{LaTeXenumerate}, and
  \texttt{LaTeXdescription}. Also, users can now redefine \verb+\makelabel+ within
  IEEE IED list controls. There may be some use for these in specialized
  applications. Thanks to Eli Barzilay for suggesting this feature.
\end{enumerate}


\section{Package Description}

The \textsf{IEEEtrantools.sty} package provides several commands from \textsf{IEEEtran.cls}
so that they can be used under other \LaTeX\ classes. This guide covers only
the differences in the use of the commands from those provided by
\textsf{IEEEtran.cls}. For complete documentation on these commands, see the relevant
sections in the \texttt{IEEEtran\_HOWTO} manual which is available at the CTAN site.

\textsf{IEEEtrantools.sty} provides \verb+\IEEEPARstart+; the \verb+\bstctlcite+ command for the
control entry types of \texttt{IEEEtran.bst} V1.00 and later; the IEEE IED list
environments; and the complete \texttt{IEEEeqnarray} family, including the
\texttt{IEEEeqnarray} support commands.

Please note that, as a package file, \textsf{IEEEtrantools.sty} will not attempt
to alter document formatting (other than the override of the IDE lists,
if the \texttt{retainorgcmds} option is not invoked) as controlled by the class
file. Therefore, there may be spacing/layout differences between the
results of the same \textsf{IEEEtran} commands under \textsf{IEEEtran.cls} and the user's
class file as different fonts, default values for the various length
commands, etc., are used than under \textsf{IEEEtran.cls}.


\section{\texttt{\textbackslash IEEEPARstart}}

\verb+\IEEEPARstart{}{}+ is used to provide a large initial ``drop cap'' letter(s) as
well as to capitialize the remaining letters of the first word of a chapter
or section (if placed within the second argument). e.\,g.,\\
\verb+\IEEEPARstart{O}{nce}+

Unlike \textsf{IEEEtran.cls}, \textsf{IEEEtrantools.sty} provides the user with a way to
control the various parameters of the \verb+\IEEEPARstart+ letters.

Below is how the user can alter them (after the \textsf{IEEEtrantools.sty} package
is loaded). Default values are shown. \textsf{IEEEtrantools} will not override
any previous definitions of these parameters if they already exist when
the package is loaded.
%
\begin{lstlisting}
% The number of lines that are indented to clear the drop cap letter.
% You may need to increase this beyond 1 + \IEEEPARstartDROPDEPTH if
% you are using lowercase letters with descenders.
\renewcommand{\IEEEPARstartDROPLINES}{2}

% The minimum number of lines left on a page to allow an \IEEEPARstart.
% Does not take into consideration glue shrink, so it tends to be overly
% cautious.
\renewcommand{\IEEEPARstartMINPAGELINES}{2}

% The height of the drop cap (*above* the baseline), is adjusted to match
% the height of this text in the current font (when \IEEEPARstart is called).
% Use a strut if you want a height not based on that of the main text font.
\renewcommand{\IEEEPARstartHEIGHTTEXT}{T}

% The depth the letter is lowered below the baseline. The height (and size)
% of the letter is determined by the sum of this value and the height of
% \IEEEPARstartHEIGHTTEXT in the current font. It is a good idea to set this
% value in terms of the baselineskip so that it can respond to changes
% therein.
\renewcommand{\IEEEPARstartDROPDEPTH}{1.1\baselineskip}

% The font the drop cap will be rendered in. The argument is a command that
% can take zero or one argument.
\renewcommand{\IEEEPARstartFONTSTYLE}{\bfseries}

% Any additional, non-font related commands needed to modify the drop cap
% letter, can take zero or one argument.
\renewcommand{\IEEEPARstartCAPSTYLE}{\MakeUppercase}

% The font that will be used to render the rest of the word (second argument
% to \IEEEPARstart), can take zero or one argument.
\renewcommand{\IEEEPARstartWORDFONTSTYLE}{\relax}

% Any additional, non-font related commands needed to modify the rest of the
% word (second argument to \IEEEPARstart), can take zero or one argument.
\renewcommand{\IEEEPARstartWORDCAPSTYLE}{\MakeUppercase}

% The horizontal separation distance from the drop letter to the main text.
% Lengths that depend on the font (i.e., ex, em, etc.) will be referenced to
% the font that is active when \IEEEPARstart is called.
\renewcommand{\IEEEPARstartSEP}{0.15em}

% The Horizontal offset applied to the left of the drop cap. The drop cap
% can be shifted left (negative) or right (positive) using this parameter.
% Lengths that depend on the font (e.g., ex, em, etc.) will be referenced
% to the font that is active when \IEEEPARstart is called.
\renewcommand{\IEEEPARstartHOFFSET}{0em}
\end{lstlisting}

For \verb+\IEEEPARstartSEP+ and \verb+\IEEEPARstartHOFFSET+, you can also reference the
length variable, \verb+\IEEEPARstartletwidth+, which will be set to the width of
the drop drop before it is rendered. e.\,g.,\\
\verb+\renewcommand{\IEEEPARstartHOFFSET}{-0.5\IEEEPARstartletwidth}+\\
will cause the drop cap to be shifted leftward by half its width.
%
\begin{lstlisting}
% Italic correction command applied at the end of the drop cap when
% evaluating its width. Without this, italic or slanted drop cap letters will
% "crash into" the main text because their full true width is not taken into
% consideration.
\renewcommand{\IEEEPARstartITLCORRECT}{\/}
\end{lstlisting}


\section{\texttt{\textbackslash bstctlcite}}

\verb+\bstctlcite{}+ is used to issue a citation for a special \texttt{IEEEtran.bst} BibTeX
style control entry which can control various operating parameters of the
\texttt{IEEEtran.bst} file (V1.00 and later):\\
\verb+\bstctlcite{IEEEexample:BSTcontrol}+

V1.2 and later of \textsf{IEEEtrantools.sty} provides for an optional argument so
that different auxiliary file specifiers may be used in documents with
multiple bibliographies:\\
\verb+\bstctlcite[@auxoutsec]{IEEEexample:BSTcontrol}+

See the \texttt{IEEEtran.bst} documentation for details:\\
\url{http://www.ctan.org/tex-archive/macros/latex/contrib/IEEEtran/bibtex}\\
\url{http://www.michaelshell.org/tex/ieeetran/bibtex/}

\verb+\bstctlcite+ operates silently and will not alter the citation numbers or
place a citation entry into the main text or bibliography (when used with
\texttt{IEEEtran.bst}). It should not be used with \texttt{.bst} files that do not provide
support for these special BST control entries.

See the \texttt{IEEEtran.bst} BibTeX style documentation for details.\\
\url{http://www.michaelshell.org/tex/ieeetran/bibtex/}\\
\url{http://www.ctan.org/tex-archive/macros/latex/contrib/IEEEtran/bibtex}


\section{\texttt{itemize}, \texttt{enumerate} and \texttt{description} (IED) lists}

\textsf{IEEEtrantools} provides revised itemize, enumerate and description list
environments that offer enhanced controls and make it much easier to
create such lists when the main text is ``block indented'' from the 
labels (IEEE style).

By default, the \LaTeX\ IED list environments are overridden with the IEEE
IED versions. To prevent this, load \textsf{IEEEtrantools.sty} with the
``\texttt{retainorgcmds}'' option:\\
\verb+\usepackage[retainorgcmds]{IEEEtrantools}+

In any event, the IEEE IED list environments are available as\\ \texttt{IEEEitemize},
\texttt{IEEEenumerate}, and \texttt{IEEEdescription}. The IEEE IED lists depend on the \LaTeX\ %
low-level \texttt{list} environment, so class files that redefine it may also alter
the IEEE IED list formatting. The original \LaTeX\ IED environments (as
provided by the \LaTeX\ kernel and class file) are always retained as
\texttt{LaTeXitemize}, \texttt{LaTeXenumerate} and \texttt{LaTeXdescription}.

Beware that the default enumerate label width will not be correct if
the class file is not using normalfont ``9)'' style labeled enumerated lists.

The support commands for the IEEE IED list environments (\verb+\IEEEsetlabelwidth+,
\verb+\IEEEusemathlabelsep+, \verb+\IEEEiedlabeljustifyl+, etc.) are also provided.


\section{The \texttt{IEEEeqnarray} family}

Please note that \textsf{IEEEtrantools} provides and sets the length variable
\verb+\IEEEnormaljot+. As IEEEtrantools is loaded, \verb+\IEEEnormaljot+ will be set to
the current value of \verb+\jot+. If the user later alters the document's nominal
\verb+\jot+ the value of \verb+\IEEEnormaljot+ should be revised as well.

The support commands for the \texttt{IEEEeqnarray} commands (\verb+\IEEEstrut+,\\ 
\verb+\IEEEeqnarrayvrule+, \verb+\IEEEvisiblestrutstrue+, etc.) are also provided.


\end{document}
